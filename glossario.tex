\makeglossaries
%**************************************************************
% Acronimi
%**************************************************************
\renewcommand{\acronymname}{Acronimi e abbreviazioni}


\newacronym [description={\glslink{apig}{Application Programming Interface}}]
    {api}{API}{Application Program Interface}

\newacronym [description={\glslink{umlg}{Unified Modeling Language}}]
    {uml}{UML}{Unified Modeling Language}

\newacronym [description={\glslink{xmlg}{eXstensible Markupd Language}}]
    {xml}{XML}{eXstensible Markupd Language}

\newacronym [description={\glslink{ucisg}{Unità Cinofile da Soccorso}}]
    {ucis}{UCIS}{Unità Cinofile da Soccorso}

\newacronym [description={\glslink{clig}{Command Line Interface}}]
{cli}{CLI}{Command Line Interface}

\newacronym{mvc}{MVC}{Model View Controller}

\newacronym{asd}{ASD}{Alternative Studio}

\newacronym{sql}{SQL}{Structured Query Language}

\newacronym [description={\glslink{ideg}{Integrated Development Environment}}]
{ide}{IDE}{Integrated Development Environment}

\newacronym [description={\glslink{cig}{Continous Integration}}]
{ci}{CI}{Continous Intengration}

\newacronym [description={\glslink{httpg}{HyperText Transfer Protocol}}]
{http}{HTTP}{HyperText Transfer Protocol}
%**************************************************************
% Glossario
%**************************************************************
\renewcommand{\glossaryname}{Glossario}

\newglossaryentry{apig}
{
  name=\glslink{api}{API}, 
  description={in informatica con il termine \emph{Application Programming Interface API} (ing. interfaccia di
    programmazione di un'applicazione) si indica ogni insieme di procedure disponibili al programmatore, di solito raggruppate a formare un
    set di strumenti specifici per l'espletamento di un determinato compito all'interno di un certo programma. La finalità è ottenere
    un'astrazione, di solito tra l'hardware e il programmatore o tra software a basso e quello ad alto livello semplificando così il lavoro
    di programmazione}
}

\newglossaryentry{umlg}
{
  name=\glslink{uml}{UML}, 
  description={in ingegneria del software \emph{UML, Unified Modeling Language} (ing. linguaggio di modellazione
    unificato) è un linguaggio di modellazione e specifica basato sul paradigma object-oriented. L'\emph{UML} svolge un'importantissima
    funzione di lingua franca nella comunità della progettazione e programmazione a oggetti. Gran parte della letteratura di settore usa
    tale linguaggio per descrivere soluzioni analitiche e progettuali in modo sintetico e comprensibile a un vasto pubblico}
}

\newglossaryentry{ucisg}
{
  name=\glslink{ucis}{UCIS}, 
  description={É un'Associazione Nazionale di Volontariato,
    inserita nell'Albo istituito presso il Dipartimento di Protezione Civile. Raggruppa, tutela e coordina i Soccorritori Cinofili presenti
    sul Territorio Nazionale}
}

\newglossaryentry{java}
{
  name=Java, 
  description={Java è un linguaggio di programmazione ad alto livello, orientato agli oggetti. Per essere il più indipendente
    possibile dalla piattaforma hardware di esecuzione si appoggia alla JVM (Java Virtual Machine) che interpreta il bytecode, cioè un file di
    output derivante dalla compilazione. Per questo viene considerato come un linguaggio semi-interpretato.
  }
}

\newglossaryentry{xmlg}
{
  name=\glslink{xml}{XML},
  description={ è un linguaggio marcatore (di markup in inglese), cioè attraverso dei tag è possibile definire e caratterizzare gli elementi,
  solitemente del testo, contenuti al loro interno. La particolarità di XML è che rende possibile creare i propri tag e per questo viene
  chiamato estensibile (eXstensible). }
}

\newglossaryentry{Git}
{
  name=Git,
  description={ È un software per il controllo di versione, nato ufficialmente nel 2005. Venne scritto inizialmente come strumento per lo
  sviluppo del kernel di Linux, sistema operativo open source molto diffuso, poi rilasciato al pubblico e allargato agli altri sistemi
  operativi, come Windows e Mac. L'interfaccia è a riga di comando (\gls{cli}) ed è sviluppato tramite C++, Python e altri linguaggi che lo
  rendono molto leggero e versatile. }
}

\newglossaryentry{Incrementale}
{
  name=incrementale,
  description={ È un concetto di Ingegneria del Software, che indica che il modello di sviluppo di un progetto viene ripetuto ad ogni
  incremento. In pratica il prodotto viene portato a una fase funzionante iniziale e da questo momento si potranno applicare incrementi,
  cioè i passi dello sviluppo software (tranne analisi, progettazione) verranno ogni volta rieseguti sul prodotto ottenuto. }
}

\newglossaryentry{Agile}
{
  name=agile,
  description={ In ingegneria del software è un insieme di modelli di sviluppo, nate in contrapposizione alla rigidità di quelli precedenti.
  Questi si basano su quattro principi fondamentali: \begin{itemize}
    \item gli individui e l'interazione è più importante dei processi e degli strumenti
    \item il software funzionante è più importante della documentazione comprensibile
    \item la collaborazione con il cliente piuttosto che la negoziazione del contratto
    \item la risposta al cambiamento è più importante di seguire il piano
  \end{itemize}}
}

\newglossaryentry{Scrum}
{
  name=Scrum,
  description={ È un metodo di sviluppo che sposa il manifesto \gls{Agile}. Questo definisce com'è composto un team e quali sono i
  comportamenti da avere all'interno di esso. Il metodo Scrum si sposa bene con il modello incrementale, questo infatti prevede degli eventi
  tra i quali lo \gls{Sprint}, il Daily Stand-up, la rethrospective, che vanno ripetuti ad ogni incremento.  }
}

\newglossaryentry{Sprint}
{
  name=Sprint,
  description={ Lo sprint è un'intervallo di tempo, solitamente di 2-4 settimane, utilizzato nel metodo \gls{Scrum} come unità di misura che
  intercorre tra una revisione e un'altra. Durante questo periodo ogni lavoro assegnato dovrà essere svolto e pronto per il successivo sprint. }
}

\newglossaryentry{Daily Stand-up}
{
  name=Daily Stand-up,
  description={ È una pratica che fa parte del metodo Scrum che prevede di fare una riunione giornaliera di massimo 15 minuti, nella quale
  ogni membro del team dice cos'ha fatto prima del meeting e pianifica cosa andrà a fare fino al successivo. }
}

\newglossaryentry{framework}
{
  name=framework,
  description={ Nello sviluppo software indica un insieme di strumenti e un'architettura a supporto del prodotto sul quale potrai andare a
  progettare e realizzare il prodotto. Spesso viene utilizzato per indicare un'insieme di librerie corredate da software che le integrano in
  maniera facilitata a ciò che le utilizza }
}

\newglossaryentry{Android}
{
  name=Android,
  description={ È un sistema operativo \gls{open source} per dispositivi mobili scritto su kernel Linux. Sviluppato da Google, esce con il primo
  smartphone nel 2007, ora è il sistema operativo mobile più diffuso al mondo. Le applicazioni installabili sono fornite tramite pacchetti
  APK, che, per i dispositivi che hanno installato i servizi Google, sono scaricabili da Play Store.}
}

\newglossaryentry{iOS}
{
  name=iOS,
  description={ È un sistema operativo proprietario di Apple Inc. per i suoi dispositivi mobili. Le applicazioni disponibili per iOS si
  possono scaricare da App Store.}
}

\newglossaryentry{frontend}
{
  name=front-end,
  description={ Nella progettazione software e sviluppo è la parte del sistema che va a interagire direttamente con l'utente. L'esempio più
  banale può essere quello dell'interfaccia grafica. }
}

\newglossaryentry{backend}
{
  name=back-end,
  description={ Nella progettazione software e sviluppo è la parte del sistema che gestisce le componenti non visibili all'utente.
  Semplificando è la logica che si occupa di elaborare i dati provenienti da un'interfaccia, producendo un output o modificando lo stato del
  sistema. }
}

\newglossaryentry{open source}
{
  name=open source,
  description={ Un software è definito open source, se il suo codice sorgente è accessibile pubblicamente. Questo comporta a sancire delle politiche di
  distribuzione che sono regolamentate da particolari licenze (MIT o Apache ad esempio). Un software open source è considerato libero, in
  contrapposizione al concetto di gratuito. Esso infatti è posseduto molto spesso da una azienda software che concede l'utilizzo, la
  modifica e la distribuzione libera regolamentate, però, dalle licenze succitate.  }
}

\newglossaryentry{Google}
{
  name=Google,
  description={ Google LLC è una delle più importanti aziende che opera nel settore informatico e offre servizi prevalentemente online. Ad
  essa è associato il nome dell'omonimo motore di ricerca dal quale la società è partita ad espandersi, diventando l'attuale colosso leader
  del settore a livello mondiale.  }
}

\newglossaryentry{tipizzato}
{
  name=Tipizzato,
  description= { Si riferisce a una caratteristica di un linguaggio di programmazione nel quale ogni programmatore deve associare ad ogni
  variabile il proprio tipo. Il compilatore associato deve garantire poi che le caratteristiche associato ad ogni tipo siano rispettate
  (tipizzazione statica).  }
}

\newglossaryentry{Issue Tracking System}
{
  name=Issue Tracking System,
  description={ Sono dei particolari software che permettono la pubblicazione, la gestione e la rimozione di issues (in italiano
  problemi/questioni). Sono utilizzati in combinazione con VCS come Git per tenere traccia di alcune criticità del software, rendendole
  visibili agli altri programmatori. }
}

\newglossaryentry{Go}
{
  name=Go,
  description={ È un linguaggio open source sviluppato da Google, che si contraddistingue per essere particolarmente efficiente nel
  risolvere problemi legati alla concorrenza.}
}


\newglossaryentry{product baseline}
{
  name=Product Baseline,
  description={ In ingegneria del software indica gli attributi attribuili a un prodotto software in un punto definito nel tempo. }
}

\newglossaryentry{diagramma di Gantt}
{
  name=diagramma di Gantt,
  description={ È una tipologia di diagramma usata principalmente nella pianificazione di progetto. L'asse orizzontale rappresenta il tempo, suddiviso in fasi e quello
  verticale le attività. Le barre nell'area del grafico descrivono la durata delle attività stesse, che si possono sovrapporre in verticale,
  indicando la possibilità di uno svolgimento parallelo di quest'ultime.}
}

\newglossaryentry{crossplatform}
{
  name=cross-platform,
  description={ Si definisce software cross-platform o multipiattaforma un applicazione implementata su più piattaforme. Il termine
  piattaforma può riferirsi all'hardware o al sistema operativo. Due piattaforme diverse possono quindi differire per l'architettura fisica
  o per quella software, come può essere nel caso di un dispositivo Apple, nel quale gira \gls{iOS}, e \gls{Android}. }
}

\newglossaryentry{web-based}
{
  name=,
  description={ Un servizio è considerabile web-based se utilizza tecnologie e linguaggi associati al mondo del web. Questi possono essere
  verificati se, ad esempio, possono essere utilizzati all'interno di un browser.}
}

\newglossaryentry{background}
{
  name=background,
  description={ In questo elaborato con il termine background si indica lo stato di un applicazione che sta eseguendo, ma che non
  può essere controllata dall'utente. Si oppone al termine foreground che definisce quando un utente esegue l'applicazione ma può
  interagire direttamente con essa.  }
}

\newglossaryentry{SDK}
{
  name=Software Development Kit,
  description={ È una raccolta di strumenti per lo sviluppo software installabile in un singolo pacchetto. Per lo sviluppo con Java, ad
  esempio, è necessario un SDK che prevede le sue librerie di base, il suo debugger e il suo compilatore. }
}


\newglossaryentry{C++}
{
  name=C++,
  description={ È l'evoluzione di C, un linguaggio ad alto livello molto utilizzato negli anni '70-'80, che introduce la programmazione
  orientata agli oggetti. Le sue caratteristiche sono le perfomance, l'efficienza e la flessibilità. Permette infatti la gestione della
  memoria a basso livello, implementado il concetto di costruttore e distruttore. Il linguaggio e completamente compilato e non possiede
  nessun layer per l'interpretazione, come ad esempio \gls{java}.}
}

\newglossaryentry{Csharp}
{
  name=C\#,
  description={ È un linguaggio orientato agli oggetti sviluppato attorno al framework .NET di Microsoft per lo sviluppo nel suo sistema
  operativo, Windows. È sviluppato per supportare la creazione di componenti e gestirne l'interazione. L'obiettivo degli sviluppatori era
  quello di creare un linguaggio che potesse girare in sistemi complessi, come sistemi operativi, ma anche in sistemi integrati di piccoli dispositivi.}
}

\newglossaryentry{clig}
{
  name=CLI,
  description={ Indica una particolare tipologia di interfaccia utente completamente a riga di comando. Solitamente le applicazioni
  utilizzano la bash (o shell testuale) del sistema operativo nel quale sta eseguendo. }
}

\newglossaryentry{database}
{
  name=database,
  description={ Base di dati in italiano, è una collezione di dati organizzati secondo un modello e una struttura. I più diffusi sono i
  database relazionali, cioè i dati sono salvati in tabelle in relazione tra di loro. }
}

\newglossaryentry{proxy}
{
  name={proxy},
  description={ In progettazione software, il proxy è un pattern architetturale. La classe che funge da proxy non è altro che un interfaccia
  per una componente che vuol essere nascosta al resto dell'architettura. }
}

\newglossaryentry{facade}
{
  name={facade},
  description={ In progettazione software, il proxy è un pattern architetturale. Tutte le funzionalità contentute in un sistema vengono
  richiamate da un'unica classe, chiamata appunto facade. Questa tecnica viene utilizzata in sistemi complessi per ridurre le dipendenze e
  modularizzare le componenti. }
}

\newglossaryentry{ideg}
{
  name=\glslink{ide}{IDE},
  description={ Si definiscono IDE ambienti di sviluppo che integrano l'editor per il codice e tutti gli strumenti per il controllo e lo
  sviluppo. Gli IDE sono studiati per facilitare la codifica con, ad esempio, suggeritori, controlli sulla sintassi, controlli statici,
  compilazione e building, debugging.  }
}

\newglossaryentry{cig}
{
  name=\glslink{ci}{Continous Integration},
  description={ In ingegneria del software, la \acrlong{ci} è una pratica che consiste nell'allineare periodicamente tutti gli spazi di
  lavoro degli sviluppatori coinvolti nello stesso progetto. La conseguenza è che ogni sviluppatore lavorerà sempre su un progetto
  aggiornato da ognuno. }
}

\newglossaryentry{nodejs}{name={Node.js},description={ È un environment per l'esecuzione runtime di JavaScript. Nel web development permette, lato server,
l'esecuzione di script in linguaggio, scaricando l'onere dal client. }}

\newglossaryentry{httpg}
{
  name=\glslink{http}{HTTP},
  description={ È un protocollo standard per architetture client-server, per l'invio e la ricezione di pagine web, ora diffuso nella
  versione HTTPS. }
}
