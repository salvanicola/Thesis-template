\makeglossaries
%**************************************************************
% Acronimi
%**************************************************************
\renewcommand{\acronymname}{Acronimi e abbreviazioni}


\newacronym [description={\glslink{apig}{Application Programming Interface}}]
    {api}{API}{Application Program Interface}

\newacronym [description={\glslink{umlg}{Unified Modeling Language}}]
    {uml}{UML}{Unified Modeling Language}

\newacronym [description={\glslink{xmlg}{eXstensible Markupd Language}}]
    {xml}{XML}{eXstensible Markupd Language}

\newacronym [description={\glslink{ucisg}{Unità Cinofile da Soccorso}}]
    {ucis}{UCIS}{Unità Cinofile da Soccorso}

\newacronym{asd}{ASD}{Alternative Studio}

%**************************************************************
% Glossario
%**************************************************************
\renewcommand{\glossaryname}{Glossario}

\newglossaryentry{apig}
{
    name=\glslink{api}{API},
    description={in informatica con il termine \emph{Application Programming Interface API} (ing. interfaccia di programmazione di un'applicazione) si indica ogni insieme di procedure disponibili al programmatore, di solito raggruppate a formare un set di strumenti specifici per l'espletamento di un determinato compito all'interno di un certo programma. La finalità è ottenere un'astrazione, di solito tra l'hardware e il programmatore o tra software a basso e quello ad alto livello semplificando così il lavoro di programmazione}
}

\newglossaryentry{umlg}
{
    name=\glslink{uml}{UML},
    description={in ingegneria del software \emph{UML, Unified Modeling Language} (ing. linguaggio di modellazione unificato) è un linguaggio di modellazione e specifica basato sul paradigma object-oriented. L'\emph{UML} svolge un'importantissima funzione di lingua franca nella comunità della progettazione e programmazione a oggetti. Gran parte della letteratura di settore usa tale linguaggio per descrivere soluzioni analitiche e progettuali in modo sintetico e comprensibile a un vasto pubblico}
}

\newglossaryentry{ucisg}
{
    name=\gls{ucis}{UCIS},
    sort=ucis,
    text=Unità Cinofile Italiane da Soccorso,
    description={É un'Associazione Nazionale di Volontariato, inserita nell'Albo istituito presso il Dipartimento di Protezione Civile. Raggruppa, tutela e coordina i Soccorritori Cinofili presenti sul Territorio Nazionale}
}

\newglossaryentry{java}
{
  name=Java,
  description={ fsdf s sadf }
}

\newglossaryentry{xmlg}
{
  name=\glslink{xml}{XML},
  description={ FGGSDF }
}

\newglossaryentry{Git}
{
  name=Git,
  description={ FGGSDF }
}

\newglossaryentry{Incrementale}
{
  name=Incrementale,
  description={ FGGSDF }
}

\newglossaryentry{Agile}
{
  name=Agile,
  description={ FGGSDF }
}

\newglossaryentry{Scrum}
{
  name=Scrum,
  description={ FGGSDF }
}

\newglossaryentry{Sprint}
{
  name=Sprint,
  description={ FGGSDF }
}

\newglossaryentry{Daily Scrum}
{
  name=Daily Scrum,
  description={ FGGSDF }
}

\newglossaryentry{Ionic}
{
  name=Ionic,
  description={ FGGSDF }
}

\newglossaryentry{framework}
{
  name=framework,
  description={ FGGSDF }
}

\newglossaryentry{Android}
{
  name=Android,
  description={ FGGSDF }
}

\newglossaryentry{iOS}
{
  name=iOS,
  description={ FGGSDF }
}

\newglossaryentry{frontend}
{
  name=frontend,
  description={ FGGSDF }
}

\newglossaryentry{open source}
{
  name=open source,
  description={ FGGSDF }
}

\newglossaryentry{Google}
{
  name=Google,
  description={ FGGSDF }
}

\newglossaryentry{tipizzato}
{
  name=Tipizzato,
  description= { fasdfa }
}

\newglossaryentry{Issue Tracking System}
{
  name=Issue Tracking System,
  description={ dfasdga }
}

\newglossaryentry{Go}
{
  name=Go,
  description={ fdasf a}
}


\newglossaryentry{product baseline}
{
  name=Product Baseline,
  description={ fdasf a}
}

\newglossaryentry{diagramma di Gantt}
{
  name=diagramma di Gantt,
  description={ fdasf a}
}

\newglossaryentry{crossplatform}
{
  name=crossplatform,
  description={f dfadsf asd }
}

\newglossaryentry{web-based}
{
  name=crossplatform,
  description={f dfadsf asd }
}

\newglossaryentry{background}
{
  name=crossplatform,
  description={f dfadsf asd }
}

\newglossaryentry{SDK}
{
  name=Software Development Kit,
  description={f dfadsf asd }
}


\newglossaryentry{C++}
{
  name=C++,
  description={f dfadsf asd }
}