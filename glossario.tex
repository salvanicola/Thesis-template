\makeglossaries
%**************************************************************
% Acronimi
%**************************************************************
\renewcommand{\acronymname}{Acronimi e abbreviazioni}
% [description={\glslink{apig}{Application Programming Interface}}]
% \newacronym
%     {api}{API}{Application Program Interface}
% [description={\glslink{umlg}{Unified Modeling Language}}]
% \newacronym
%     {uml}{UML}{Unified Modeling Language}
% [description={\glslink{ucisg}{Unità Cinofile da Soccorso}}]
% \newacronym
%     {ucis}{UCIS}{Unità Cinofile da Soccorso}
% [description={\glslink{xmlg}{eXstensible Markupd Language}}]
% \newacronym
%     {xml}{XML}{eXstensible Markupd Language}
\newacronym{big}{BIG}{Best Intelligent Ground}

%**************************************************************
% Glossario
%**************************************************************
\renewcommand{\glossaryname}{Glossario}

\newglossaryentry{api}
{
    name=api,
    description={in informatica con il termine \emph{Application Programming Interface API} (ing. interfaccia di programmazione di un'applicazione) si indica ogni insieme di procedure disponibili al programmatore, di solito raggruppate a formare un set di strumenti specifici per l'espletamento di un determinato compito all'interno di un certo programma. La finalità è ottenere un'astrazione, di solito tra l'hardware e il programmatore o tra software a basso e quello ad alto livello semplificando così il lavoro di programmazione}
}

\newglossaryentry{uml}
{
    name=uml,
    description={in ingegneria del software \emph{UML, Unified Modeling Language} (ing. linguaggio di modellazione unificato) è un linguaggio di modellazione e specifica basato sul paradigma object-oriented. L'\emph{UML} svolge un'importantissima funzione di lingua franca nella comunità della progettazione e programmazione a oggetti. Gran parte della letteratura di settore usa tale linguaggio per descrivere soluzioni analitiche e progettuali in modo sintetico e comprensibile a un vasto pubblico}
}

\newglossaryentry{ucis}
{
    name=ucis,
    sort=ucis,
    text=Unità Cinofile Italiane da Soccorso,
    description={É un'Associazione Nazionale di Volontariato, inserita nell'Albo istituito presso il Dipartimento di Protezione Civile. Raggruppa, tutela e coordina i Soccorritori Cinofili presenti sul Territorio Nazionale}
}

\newglossaryentry{UCIS Report Tool}
{
  name=UCIS Report Tool,
  description={Applicazione in dotazione attualmente a UCIS sviluppata 5 anni fa e pubblicata da Alternative Studio nel 2018}
}

\newglossaryentry{Java}
{
  name=Java,
  description={ fsdf s sadf }
}

\newglossaryentry{xml}
{
  name=xml,
  description={ FGGSDF }
}
