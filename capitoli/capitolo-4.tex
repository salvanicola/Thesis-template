% !TEX encoding = UTF-8
% !TEX TS-program = pdflatex
% !TEX root = ../tesi.tex

%**************************************************************
\chapter{Conclusioni}
\label{cap:valutazione-retrospettiva}
%**************************************************************

% \intro{Breve introduzione al capitolo}

\section{Obiettivi raggiunti}

Prima di cominciare lo stage assieme al relatore \myProf e al mio tutor aziendale Francesco Costa è stato messo a punto
un Piano di Lavoro, cioè un documento dove si è definito, in linea generale, gli obiettivi e la pianificazione dello
stage. Di seguito la lista degli obiettivi programmati trattati in esso: 

\renewcommand{\arraystretch}{2}
\begin{longtable}{|p{10cm}|p{4cm}|}%
  \caption{Tabella degli obiettivi} 
  \label{tab:obiettivi} \\
    \hline
    \textbf{Obiettivo} & \textbf{Risultato} \\
    \hline
    \endhead
    \multicolumn{2}{|l|}{\textbf{Obbligatori}} \\ \hline
    Progettazione e realizzazione di applicazione mobile                      & Soddisfatto \\ \hline
    Implementazione servizio di geolocalizzazione preciso nell’applicazione   & Soddisfatto \\ \hline
    Versione beta dell'applicazione da pubblicare sullo store                 & Soddisfatto \\ \hline
    \multicolumn{2}{|l|}{\textbf{Desiderabili}} \\ \hline
    Applicazione cross-platform, combatibile anche su dispositivi datati      & Soddisfatto \\ \hline
    Messa in produzione dell’applicazione                                     & Soddisfatto \\ \hline
    \multicolumn{2}{|l|}{\textbf{Facoltativi}} \\ \hline
    Implementazione delle notifiche push                                      & Non Soddisfatto \\ \hline
    Integrazione di una chat                                                  & Non Soddisfatto \\ \hline
    Applicazione funzionante anche su dispositivi senza servizi Google        & Non Soddisfatto \\ \hline
\end{longtable}%

\noindent In generale sono stati rispettati tutti gli obiettivi, tranne quelli ritenuti facoltativi. Quest'ultimi posso
dire di averli trattati solo parzialmente e non averli completati. Durante lo stage ho infatti studiato i metodi per le
notifiche push su \gls{Android} e su come sviluppare applicazioni in dispositivi senza servizi Google, come ad esempio
alcuni modelli recenti \textit{Huawei}. Non nascondo un po' di rammarico nel non essere riuscito a completare tutti gli
obiettivi, ma a posteriori posso dire che forse nel piano di lavoro siamo stati un po' troppo ambiziosi e non si sono
considerati tutti i rischi. Per questo quelli facoltativi sono stati posti nell'ultima settimana, dando ovvia priorità
agli altri. Considerando ciò posso comunque ritenermi soddisfatto del risultato ottenuto. Credo che parte il
successo sia dovuto al metodo \gls{Agile} integrato nel nostro metodo di lavoro che ci ha permesso di settimana in
settimana di definire chiaramente cosa sarei andato a fare, correggendo il piano se qualcosa avesse richiesto più tempo. 

\subsection{Valutazione della scelta}

Per capire correttamente, perchè ho scelto la proposta di Alternative Stuidio, è bene specificare alcune cose. \\
\noindent Quello di cui ero alla ricerca era un contesto piccolo con ampi margini di crescita e di formazione personale
sulle tecnologie moderne, più che sul metodo di lavoro, che invece tende a essere molto più definito in aziende grandi,
con un modello affermato e utilizzato da tempo. Quello che mi interessava era sperimentare vari ruoli, comprendere le
varie responsabilità che questi comportano e come interfacciarsi con altri dipendenti. \\
Oltre al contesto aziendale ero attratto anche dall'argomento del progetto. Volevo sperimentare nuove tecnologie e nuovi linguaggi,
disegnare e creare del software, piuttosto che mantenerlo o estenderlo. Da questo punto di vista l'Università di Padova,
attraverso Stage-IT mi ha messo in contatto con molte aziende interessanti. Inoltre la mia ricerca del progetto di stage si è basata sul desiderio
di lavorare in ambito frontend e questo ha ristretto di molto le varie opportunità. \\
Il fattore decisivo sulla scelta dello stage è stato il fatto che da tempo volevo sperimentare le tecnologie per lo sviluppo di applicazioni
mobile. Io stesso, precedentemente, avevo provato ad avviare qualche piccolo progetto in questo campo e per questo ho voluto
scegliere una proposta che avesse questo come tema principale. \\
In questo senso l'offerta di \acrlong{asd} ha subito catturato la mia attenzione e nonostante fosse la loro prima esperienza con uno
studente laureando, ho preso contatti per avviare con loro il mio stage.

\subsubsection{Obiettivi personali}
\label{sec:obiettivi-pers}
Le esperienze lavorative precedenti a questa mi avevano insegnato cosa vuol dire lavorare in una squadra e cos'è prendersi carico
di responsabilità, ma con nessuna di queste avevo messo in pratica ciò che avevo studiato finora. Sentivo quindi il bisogno di consolidare
il bagaglio di nozioni acquisite durante gli anni del corso e lo stage ne è stata l'occasione perfetta. Certo, ci sono stati i molteplici
progetti didattici, ma nulla è come entrare nel contesto lavorativo e avere contatto diretto con questa realtà. Di fatto
lo stage ha unito due mondi che già conoscevo, quello lavorativo, ma sentendomi sicuro di me stesso, sapendo di essere
qualificato per affrontare quello che mi aspettava. \\
Riguardo lo stage in generale avevo aspettative simili che si possono riassumere nei seguenti punti.
\begin{itemize}
	\item collaborare con colleghi con molta più esperienza e conoscenza di me;
	\item imparare nuove tecnologie e distinguere quelle buone da quelle cattive;
	\item migliorare la mia gestione del tempo e organizzarmi con le scadenze dei vari compiti;
	\item trovare proposte e contatti per progetti futuri.
\end{itemize}

Da questo punto di vista, a posteriori, credo sia stato un successo. Lo stage è riuscito a non deludermi e a soddisfare
tutti gli obiettivi in maniera esaustiva. Di seguito spiego in che modo:

\begin{itemize}
    \item Alternative Studio mi ha dato la possibilità di collaborare a stretto contatto con professionisti nel settore che mi hanno formato
    e hanno plasmato il mio metodo di lavoro. Sono stato contento di aver lavorato in un contesto giovane e in espansione, dove un giovane
    stagista può sentirsi a proprio agio crescendo assieme alla realtà in cui è immerso. Nonostante questo mi sento di aver imparato a
    distinguere ciò che è \textit{professionale} da ciò che non lo è.
    \item Nella sezione successiva descriverò in dettaglio quali tecnologie ho imparato. In generale penso che la lunga analisi avviata
    nello studio di fattibilità mi abbia aiutato e fatto capire cosa mi propone il mondo informatico. Sono riuscito a sviluppare un metodo
    per esaminare e sviscerare le caratteristiche di ognuna delle tecnologie, che consiste nelle leggere la documentazione ufficiale,
    costruire una mia opinione e solo in seguito leggere opinioni di altri, cercando ovviamente le più autorevoli. Ho capito anche che un
    buon software non è composto solo da una buona codifica, una buona interfaccia o un buon supporto, ma da una grande community che
    fornisce un troubleshooting adeguato a tutti i problemi che puoi incorrere.
    \item Sono abbastanza soddisfatto di come sia riuscito a gestire il tempo. Ero molto preoccupato che la gestione elastica dettata dal
    telelavoro mi avrebbe penalizzato, date le distrazioni "casalinghe", o al contrario non avere orari di ufficio mi avrebbe portato a fare
    più ore di quelle stabilite. Non nascondo che è ovviamente capitato, quando uno sviluppatore si intestardisce nel trovare una
    soluzione a un problema può rimane anche ore su di esso, ma spesso il tutor è intervenuto a regolarizzare il ritmo lavorativo. 
    \item Grazie a Stage-IT sono riuscito a raccogliere molti contatti. Mi piacerebbe continuare a collaborare con \acrlong{asd} nei suo
    progetti futuri.
\end{itemize}

\section{Valutazione degli strumenti utilizzati}

Gli strumenti che mi ha messo a disposizione \acrlong{asd} sono stato un notevole supporto al mio lavoro.

\paragraph{Gitlab} Avendo avuto esperienza con altri software, come Github, posso dire che mi è sembrato completo in
tutte le sue funzionalità. Il sistema di issues è stato utilissimo a notificare e salvare idee. Purtroppo alcune delle
sue parti non sono state utilizzate a dovere durante il progetto, come quella che si occupa della \gls{ci} o le
\textit{operations}, piccoli script che vengono eseguiti a ogni push, che possono calcolare, ad esempio alcuni dei
valori di verifica del codice. 

\paragraph{Codifica} WebStorm è un'\gls{ide} potentissimo, completo sotto ogni punto di vista. Ha moltissime
funzionalità e supporta moltissimi linguaggi. Tuttavia l'ho trovato molto pesante per le risorse del computer,
soprattutto la RAM, e con molti errori e bug, che spesso interrompono il workflow. 

\paragraph{Postman} Il software è risultato molte utile e perfetto per lo scopo. Mi ha permesso di testare le API, prima
di procedere con la codifica. La creazione di un \textit{environment}, cioè una collezione di richeste salvate e
ordinate, si è rivelato ottimo per la collaborazione con gli altri componenti del team.

\section{Valutazione delle tecnologie scelte}
\paragraph{Ionic} Sono rimasto molto soddisfatto dal framework scelto. Il supporto e le soluzioni proposte della community sono
sempre state adeguate al problema. Ci sono alcuni punti negativi, quali una documentazione a tratti troppo sintetica o
incompleta e la scelta limitata dei plugin nativi disponibili. Questo ci ha portato durante il progetto a sviluppare un
plugin apposito per la gestione degli account tramite il manager a disposizione dei dispositivi Android. In generale
l'effettiva versatilità del sistema e la disponibilità di librerie grafiche moderne e piacevoli sono state motivo per
apprezzare sempre di più questo potente framework.

\paragraph{Angular} È stato un eccellente piattaforma nel quale sviluppare l'applicazione. Ho avuto molte difficoltà nel
gestire le risorse asincrone, penso che esistano sistemi meno macchinosi. Ho apprezzato invece l'architettura che
garantisce una modularità quasi obbligata e quindi uno sviluppo eccellente da parte di un team.

\paragraph{Plugin utilizzati}
Nonostante Ionic offrisse dei plugin cosiddetti \textit{nativi}, questi non erano altro che adattatori per Capacitor, di plugin
sviluppati per Cordova. Sono rimasto a lungo a leggere la loro documentazione che spesso non si è rivelata
professionale, si parla, comunque di progetti open source. Non nego inoltre di essere spesso andato a leggere il loro
codice per capire come dovessero essere utilizzati. Tuttavia questa scelta è stata fatta in maniera quasi obbligata date
le risorse a disposizione del progetto. 

\section{Considerazioni sul prodotto}
Quello che ho ottenuto alla fine del progetto, non è certamente un prodotto che può considerarsi completo. Rispetta
effettivamente i requisiti imposti inizialmente, ma possiede ancora alcune imperfezioni che sicuramente verrano corrette
con il tempo. L'applicazione non è ancora stata resa disponibile ai membri di \gls{ucis}, ma è stata testata da alcuni.
Le impressioni sono ottime ed è saltato all'occhio il confronto con l'applicazione precedente. È vero che l'impatto su
un utente inesperto di un pacchetto grafico moderno e accattivante, è sicuramente forte, ma sono stati notate le
soluzioni ai problemi dell'applicazione precedente. \\
\noindent Sono state definite comunque delle posibili estensioni al progetto. Si è programmato di portare a termine gli
obiettivi facoltativi, implementazione delle notifiche push, integrazione della chat e l'applicazione funzionante senza
serivizi Google. Inoltre si stanno prendendo accordi per sviluppare la versione iOS, alla quale spero di partecipare.


\section{Conoscenze e competenze acquisite}
\subsection{Collaborazione con i colleghi}
Il lavoro di squadra era uno dei punti che più mi preoccupava. Prima dello stage ero convinto che iniziare un progetto in più persone senza
dividere bene i ruoli e le mansioni in modo definito fosse impossibile. Dopo questa esperienza mi sento di dire che lavorare in un team
cercando il confronto continuo e avendo sempre qualcuno a supporto è il modo migliore per raggiungere gli obiettivi. \\ 
\noindent Ho capito che il fallimento è contemplato in questo settore e che se una cosa non riesce puoi sempre rivolgerti a qualcun altro
che ti darà una visione diversa del problema. Questo approccio mi è tornato utile molto spesso durante il progetto quando mi trovavo fermo
in punto, senza trovare vie d'uscita, rivolgermi al tutor è sempre stata la soluzione migliore.
\noindent Un'altro aspetto al quale tengo molto e la gestione delle relazioni tra colleghi che deve andare oltre al semplice rapporto
lavorativo, per creare un ambiente rilassato e confortevole, dove è più semplice lavorare. Ho imparato quindi a essere professionale, ma al
contempo non troppo rigido.

\subsection{Analisi e Studio di Fattibilità}
Prima di intraprendere il percorso di stage la consideravo una fase noiosa e poco utile di un progetto. Ho completamente cambiato idea
affrontandola con il tutor. In precedenza, probabilmente avevo un approccio meno interattivo e un po' retrogrado, ma con il tutor sono
riuscito a cambiarlo radicalmente. \\ 
\noindent Dato che i requisiti erano già stati quasi tutti fissati, lo studio delle tecnologie da utilizzare è quello che mi ha coinvolto di
più. Capire le caratteristiche di ognuna di esse e apprezzarne le differenze mi ha appassionato portandomi ad apprezzare questo processo
che prima ritenevo superfluo. È da sottolineare che questo progetto partendo da zero, aveva bisogno di un analisi completa e questo mi ha
dato molta libertà nella ricerca. 

\subsection{Tecnologie apprese}
Oltre ad affinare l'utilizzo di Git e degli strumenti di controllo di versione, ho appreso meglio il concetto di \gls{ci}
attraverso gli strumenti di GitLab. Ho conosciuto il mondo dello sviluppo mobile e di tutte le tecnologie che ci stanno attorno, ordinando
le conoscenze acquisite in precedenza. È stato affascinante passare da utente di applicazioni per smartphone a programmatore, iniziando a
notare tutti gli stessi errori e le imperfezioni che altri sviluppatori hanno commesso come me su alcune cose. \\
\noindent Ho imparato a programmare con Ionic e Angular, che sicuramente mi saranno utili in futuro nel mondo del web development. Ora sono
in grado di gestire in modo efficiente la programmazione concorrente attraverso le librerie di \gls{nodejs} con le \textit{Promise}. So
interfacciarmi con delle API su un server web. Ho imparato, inoltre, a avviare un progetto in Android tramite il framework nativo e
modificare le sue parti in \acrshort{xml} e \gls{java}.

\section{Valutazioni personali}
Ritengo di aver avuto parecchie aspettative sullo stage finale, considerandolo la conclusione di un percorso che mi avrebbe lanciato poi nel
mondo del lavoro e testando tutte le mie conoscenze apprese durante gli studi. Ho scoperto però che non è stato del tutto così. Avrei dovuto
affrontarlo con aspettative più basse e un altro tipo di mentalità. Non è stata un'esperienza negativa, al contrario ho imparato molto e
credo sia quella più utile proposta dal corso di laurea, ma da questa ho scoperto che mi mancano ancora moltissime competenze e conoscenze
per essere del tutto pronto al mondo del lavoro. \\
\noindent Nel complesso mi sento soddisfatto e credo di aver fatto un ottimo lavoro, mettendo tutto me stesso nel progetto e impegnandomi
nella sua realizzazione. Grazie al contesto aziendale nel quale mi sono trovato ho avuto tutti gli strumenti per crescere e spero di
trovarne di simili nel prossimo futuro. So che dovrò spendere molto più delle 300 ore messe a disposizione dallo stage per sentirmi
completamente realizzato e pronto per diventare un vero e proprio informatico.