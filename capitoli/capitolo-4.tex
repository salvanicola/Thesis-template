% !TEX encoding = UTF-8
% !TEX TS-program = pdflatex
% !TEX root = ../tesi.tex

%**************************************************************
\chapter{Valutazione Retrospettiva}
\label{cap:valutazione-retrospettiva}
%**************************************************************

% \intro{Breve introduzione al capitolo}

\section{Obiettivi raggiunti}
\subsection{Obiettivi dello stage}
In questa sezione analizzerò quali degli obiettivi descritti nel \hyperref[sec:obiettivi]{capitolo 2} sono stati raggiunti e quali no, dando
una giustificazione nel caso negativo.

\renewcommand{\arraystretch}{2}
\begin{longtable}{|p{10cm}|p{4cm}|}%
  \caption{Tabella degli obiettivi} 
  \label{tab:obiettivi} \\
    \hline
    \textbf{Obiettivo} & \textbf{Risultato} \\
    \hline
    \endhead
    \multicolumn{2}{|l|}{\textbf{Obbligatori}} \\ \hline
    Progettazione e realizzazione di applicazione mobile                      & Soddisfatto \\ \hline
    Implementazione servizio di geolocalizzazione preciso nell’applicazione   & Soddisfatto \\ \hline
    Versione beta dell'applicazione da pubblicare sullo store                 & Soddisfatto \\ \hline
    \multicolumn{2}{|l|}{\textbf{Desiderabili}} \\ \hline
    Applicazione cross-platform, combatibile anche su dispositivi datati      & Soddisfatto \\ \hline
    Messa in produzione dell’applicazione                                     & Soddisfatto \\ \hline
    \multicolumn{2}{|l|}{\textbf{Facoltativi}} \\ \hline
    Implementazione delle notifiche push                                      & Non Soddisfatto \\ \hline
    Integrazione di una chat                                                  & Non Soddisfatto \\ \hline
    Applicazione funzionante anche su dispositivi senza servizi Google        & Non Soddisfatto \\ \hline
\end{longtable}%

\noindent In generale sono stati rispettati tutti gli obiettivi, tranne quelli ritenuti facoltativi. Questi non erano inseriti nel piano di lavoro, se
non nell'ultima settimana in caso l'applicazione fosse stata già pronta. In ogni caso gli obiettivi facoltativi inseriti fanno parte dei requisiti
dell'applicazione e per questo non sono stati esclusi completamente. Durante lo stage ho infatti studiato i metodi per le notifiche push su
\gls{Android} e su come sviluppare applicazioni in dispositivi senza servizi Google, come ad esempio alcuni modelli recenti \textit{Huawei}.
\\
\noindent Grazie al metodo di lavoro utilizzato, ho rispettato anche il piano di lavoro concordato con il tutor aziendale. Le brevi riunioni quasi
giornaliere durante le quali si definivano le piccole task, sono state molto utili allo scopo, velocizzando lo sviluppo e razionalizzando il
tempo a disposizione per ognuna di esse. 

\subsection{Obiettivi personali}
Questo stage aveva anche alcuni obiettivi personali descritti nella sezione \hyperref[sec:obiettivi-pers]{2.1.2}:
\begin{itemize}
    \item Alternative Studio mi ha dato la possibilità di collaborare a stretto contatto con professionisti nel settore che mi hanno formato
    e hanno plasmato il mio metodo di lavoro. Sono stato contento di aver lavorato in un contesto giovane e in espansione, dove un giovane
    stagista può sentirsi a proprio agio crescendo assieme alla realtà in cui è immerso. Nonostante questo mi sento di aver imparato a
    distinguere ciò che è \textit{professionale} da ciò che non lo è.
    \item Nella sezione successiva descriverò in dettaglio quali tecnologie ho imparato. In generale penso che la lunga analisi avviata
    nello studio di fattibilità mi abbia aiutato e fatto capire cosa mi propone il mondo informatico. Sono riuscito a sviluppare un metodo
    per esaminare e sviscerare le caratteristiche di ognuna delle tecnologie, che consiste nelle leggere la documentazione ufficiale,
    costruire una mia opinione e solo in seguito leggere opinioni di altri, cercando ovviamente le più autorevoli. Ho capito anche che un
    buon software non è composto solo da una buona codifica, una buona interfaccia o un buon supporto, ma da una grande community che
    fornisce un troubleshooting adeguato a tutti i problemi che puoi incorrere.
    \item Sono abbastanza soddisfatto di come sia riuscito a gestire il tempo. Ero molto preoccupato che la gestione elastica dettata dal
    telelavoro mi avrebbe penalizzato, date le distrazioni "casalinghe", o al contrario non avere orari di ufficio mi avrebbe portato a fare
    più ore di quelle stabilite. Non nascondo che è ovviamente capitato, quando uno sviluppatore si intestardisce nel trovare una
    soluzione a un problema può rimane anche ore su di esso, ma spesso il tutor è intervenuto a regolarizzare il ritmo lavorativo. 
    \item Grazie a Stage-IT sono riuscito a raccogliere molti contatti. Mi piacerebbe continuare a collaborare con \acrlong{asd} nei suo
    progetti futuri.
\end{itemize}

\section{Conoscenze e competenze acquisite}
\subsection{Collaborazione con i colleghi}
Il lavoro di squadra era uno dei punti che più mi preoccupava. Prima dello stage ero convinto che iniziare un progetto in più persone senza
dividere bene i ruoli e le mansioni in modo definito fosse impossibile. Dopo questa esperienza mi sento di dire che lavorare in un team
cercando il confronto continuo e avendo sempre qualcuno a supporto è il modo migliore per raggiungere gli obiettivi. \\ 
\noindent Ho capito che il fallimento è contemplato in questo settore e che se una cosa non riesce puoi sempre rivolgerti a qualcun altro
che ti darà una visione diversa del problema. Questo approccio mi è tornato utile molto spesso durante il progetto quando mi trovavo fermo
in punto, senza trovare vie d'uscita, rivolgermi al tutor è sempre stata la soluzione migliore.
\noindent Un'altro aspetto al quale tengo molto e la gestione delle relazioni tra colleghi che deve andare oltre al semplice rapporto
lavorativo, per creare un ambiente rilassato e confortevole, dove è più semplice lavorare. Ho imparato quindi a essere professionale, ma al
contempo non troppo rigido.

\subsection{Analisi e Studio di Fattibilità}
Prima di intraprendere il percorso di stage la consideravo una fase noiosa e poco utile di un progetto. Ho completamente cambiato idea
affrontandola con il tutor. In precedenza, probabilmente avevo un approccio meno interattivo e un po' retrogrado, ma con il tutor sono
riuscito a cambiarlo radicalmente. \\ 
\noindent Dato che i requisiti erano già stati quasi tutti fissati, lo studio delle tecnologie da utilizzare è quello che mi ha coinvolto di
più. Capire le caratteristiche di ognuna di esse e apprezzarne le differenze mi ha appassionato portandomi ad apprezzare questo processo
che prima ritenevo superfluo. È da sottolineare che questo progetto partendo da zero, aveva bisogno di un analisi completa e questo mi ha
dato molta libertà nella ricerca. 

\subsection{Tecnologie apprese}
Oltre ad affinare l'utilizzo di Git e degli strumenti di controllo di versione, ho appreso meglio il concetto di \gls{ci}
attraverso gli strumenti di GitLab. Ho conosciuto il mondo dello sviluppo mobile e di tutte le tecnologie che ci stanno attorno, ordinando
le conoscenze acquisite in precedenza. È stato affascinante passare da utente di applicazioni per smartphone a programmatore, iniziando a
notare tutti gli stessi errori e le imperfezioni che altri sviluppatori hanno commesso come me su alcune cose. \\
\noindent Ho imparato a programmare con Ionic e Angular, che sicuramente mi saranno utili in futuro nel mondo del web development. Ora sono
in grado di gestire in modo efficiente la programmazione concorrente attraverso le librerie di \gls{nodejs} con le \textit{Promise}. So
interfacciarmi con delle API su un server web. Ho imparato, inoltre, a avviare un progetto in Android tramite il framework nativo e
modificare le sue parti in \acrshort{xml} e \gls{java}.

\section{Valutazioni personali}
Ritengo di aver avuto parecchie aspettative sullo stage finale, considerandolo la conclusione di un percorso che mi avrebbe lanciato poi nel
mondo del lavoro e testando tutte le mie conoscenze apprese durante gli studi. Ho scoperto però che non è stato del tutto così. Avrei dovuto
affrontarlo con aspettative più basse e un altro tipo di mentalità. Non è stata un'esperienza negativa, al contrario ho imparato molto e
credo sia quella più utile proposta dal corso di laurea, ma da questa ho scoperto che mi mancano ancora moltissime competenze e conoscenze
per essere del tutto pronto al mondo del lavoro. \\
\noindent Nel complesso mi sento soddisfatto e credo di aver fatto un ottimo lavoro, mettendo tutto me stesso nel progetto e impegnandomi
nella sua realizzazione. Grazie al contesto aziendale nel quale mi sono trovato ho avuto tutti gli strumenti per crescere e spero di
trovarne di simili nel prossimo futuro. So che dovrò spendere molto più delle 300 ore messe a disposizione dallo stage per sentirmi
completamente realizzato e pronto per diventare un vero e proprio informatico.