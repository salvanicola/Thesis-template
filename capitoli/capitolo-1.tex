% !TEX encoding = UTF-8
% !TEX TS-program = pdflatex
% !TEX root = ../tesi.tex

%**************************************************************
\chapter{Introduzione}
\label{cap:introduzione}
%**************************************************************
\section{Convenzioni tipografiche}

Riguardo la stesura del testo, relativamente al documento sono state adottate le seguenti convenzioni tipografiche:
\begin{itemize}
	\item gli acronimi, le abbreviazioni e i termini ambigui o di uso non comune menzionati vengono definiti nel glossario, situato alla fine del presente documento;
	\item per la prima occorrenza dei termini riportati nel glossario viene utilizzata la seguente nomenclatura: \emph{parola}\glsfirstoccur;
	\item i termini in lingua straniera o facenti parti del gergo tecnico sono evidenziati con il carattere \emph{corsivo}.
\end{itemize}

\section{Scopo del documento}
Qui devo scrivere i contenuti di ciò che dirò.

%**************************************************************

\subsection{Organizzazione del testo}

\begin{description}
    \item[{\hyperref[cap:processi-metodologie]{Il secondo capitolo}}] descrive ...

    \item[{\hyperref[cap:descrizione-stage]{Il terzo capitolo}}] approfondisce ...

    \item[{\hyperref[cap:analisi-requisiti]{Il quarto capitolo}}] approfondisce ...

    \item[{\hyperref[cap:progettazione-codifica]{Il quinto capitolo}}] approfondisce ...

    \item[{\hyperref[cap:verifica-validazione]{Il sesto capitolo}}] approfondisce ...

    \item[{\hyperref[cap:conclusioni]{Nel settimo capitolo}}] descrive ...
\end{description}

% Introduzione al contesto applicativo.\\
% \noindent Esempio di utilizzo di un termine nel glossario \\
% \gls{api}. \\
% \noindent Esempio di citazione in linea \\
% \cite{site:agile-manifesto}. \\
% \noindent Esempio di citazione nel pie' di pagina \\
% citazione\footcite{womak:lean-thinking} \\

%**************************************************************
\section{L'azienda}

\begin{figure}[htbp]
\begin{center}
\includegraphics[height=6cm]{app_logo}
\end{center}
\end{figure}

Alternative Studio è una web agency che fornisce soluzioni professionali su misura, costruite secondo le esigenze del cliente. Si occupa principalmente di sviluppo web e marketing. È un'azienda piccola che raccoglie poche risorse umane, ma molte energie che continuano a spingere per crescere. Opera da appena sei anni nel settore del web development, ma ha abbracciato anche altre iniziative, collaborando in progetti più grandi con altre realtà.
Negli ultimi anni l'azienda si è cimentata nello sviluppo di un gestionale per l'organizzazione \gls{ucis}. Quest ultimo possiede un'\gls{api} volta alla ricezione e all'elaborazione di attività registrate durante addestramento, soccorso o esercitazioni.

%**************************************************************
\section{L'idea}

L'attuale applicazione per smartphone che si occupa di registrare le attività succitate è stata sviluppata molti anni fa ed essendo molto instabile è nata l'esigenza di un refactoring di quest'ultima. Volendo utilizzare tecnologie moderne per lo sviluppo mobile e crossplatform è nata l'esigenza di un'indagine preventiva sulle tecnologie da utilizzare. \\
Alternative Studio ha pensato, quindi, di aprire una posizione perfetta per un laureando, che cerchi un progetto formante, che lo metta continuamente alla prova.

%**************************************************************

\section{Metodo di Lavoro}

Dato le contenute risorse umane a disposizione Alternative Studio adotta un ciclo di sviluppo software \gls{Incrementale} con qualche introduzione di processi da quello \gls{Agile}, in particolared dal metodo \gls{Scrum}. Durante lo stage lo studente è stato incaricato anche di introdurre qualche concetto di questi cicli di vita all'interno del contesto aziendale, come quello di \gls{Sprint} e di \gls{Daily Scrum}. \\
\noindent A supporto dello sviluppo del progetto sono stati utilizzati i seguenti strumenti:
\begin{itemize}
	\item Gitlab: software open source per la gestione di repository \gls{Git} e supporto alla Continous Integration.
	\begin{figure}[htbp]
	\begin{center}
	\includegraphics[height=3cm]{gitlab}
	\end{center}
	\end{figure}
  Utile anche per l'assegnazione di piccole tasks e fissare scadenze, milestone e monitoraggio delle statistiche del progetto.
  \item Slack:
  \item Telegram:
  \item WebStorm: potente IDE, modulare, che con l'aggiunta di plugin acquista funzionalità, come il conteggio delle ore di programmazione, scadenze e gestione del repository \gls{Git} locale.
  \begin{figure}[htbp]
  \begin{center}
  \includegraphics[height=3cm]{webstorm}
  \end{center}
  \end{figure}
\end{itemize}

%**************************************************************

\section{Tecnologie utilizzate}

Lo stage si è focalizzato sull'utilizzo di tecnologie mobile per lo sviluppo di applicazioni per smartphone. Queste sono spesso nuove e in continua evoluzione, è per questo che Alternative Studio ha avviato uno studio approfondito per scegliere quelle adatte allo sviluppo della propria applicazione. 

\subsection{Ionic}

\gls{Ionic} è un \gls{framework} open source che fornisce strumenti agli sviluppatori, come librerie grafiche e plugins nativi per dialogare con \gls{Android} e \gls{iOS}. Esso permette lo sviluppo mobile utilizzando tecnologie standard web come Angular, React e Vue, evitando lo sviluppo in linguaggio nativo dei singoli sistemi operativi.
Questo è reso possibili da wrapper esistenti su ogni piattaforma volti a eseguire l'applicazione.

\subsection{Angular}

Come framework web per lo sviluppo \gls{frontend} si è utilizzato Angular. Nonostante l'esperenzia di


%**************************************************************
