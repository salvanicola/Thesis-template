% !TEX encoding = UTF-8
% !TEX TS-program = pdflatex
% !TEX root = ../tesi.tex

%**************************************************************
\chapter{Introduzione}
\label{cap:introduzione}
%**************************************************************
\section{Descrizione dello stage}

La scelta dello stage è stata molto difficile. Ero alla ricerca, infatti, di un contesto specifico per la mia prima esperienza lavorativa
nel settore informatico. Consultando tutte le varie proposte, durante l'evento Stage-IT, ho fatto fatica a trovare una proposta che mi
convincesse a fondo. Alcune delle aziende che ho contattato si sono dimostrate disponibili nei miei confronti, ma per le
tempistiche legate alla laura ho dovuto rinunciare alle loro proposte.
%  Tra le motivazioni per le quali ho scelto \gls{asd} c'è senz'altro un progetto chiaro e
% stimolante, ma anche la disponibilità immediata allo stage.

\subsection{Motivazioni della scelta}

Quello di cui ero alla ricerca era un contesto piccolo con ampi margini di crescita e di formazione personale sulle tecnologie moderne, più
che sul metodo di lavoro, che invece tende a essere molto più definito in aziende grandi, con un modello affermato e utilizzato da tempo.
Quello che mi interessava era sperimentare vari ruoli, comprendere le varie responsabilità che questi comportano e come interfacciarsi con
altri dipendenti. \\
Oltre al contesto aziendale ero attratto anche dall'argomento del progetto. Volevo sperimentare nuove tecnologie e nuovi linguaggi,
disegnare e creare del software, piuttosto che mantenerlo o estenderlo. Da questo punto di vista l'Università di Padova,
attraverso Stage-IT mi ha messo in contatto con molte aziende interessanti. Inoltre la mia ricerca del progetto di stage si è basata sul desiderio
di lavorare in ambito frontend e questo ha ristretto di molto le varie opportunità. \\
Il fattore decisivo sulla scelta dello stage è stato il fatto che da tempo volevo sperimentare le tecnologie per lo sviluppo di applicazioni
mobile. Io stesso, precedentemente, avevo provato ad avviare qualche piccolo progetto in questo campo e per questo ho voluto
scegliere una proposta che avesse questo come tema principale. \\
In questo senso l'offerta di \acrlong{asd} ha subito catturato la mia attenzione e nonostante fosse la loro prima esperienza con uno
studente laureando, ho preso contatti per avviare con loro il mio stage.

\subsection{Obiettivi personali}
\label{sec:obiettivi-pers}
Le esperienze lavorative precedenti a questa mi avevano insegnato cosa vuol dire lavorare in una squadra e cos'è prendersi carico
di responsabilità, ma con nessuna di queste avevo messo in pratica ciò che avevo studiato finora. Sentivo quindi il bisogno di consolidare
il bagaglio di nozioni acquisite durante gli anni del corso e lo stage ne è stata l'occasione perfetta. Certo, ci sono stati i molteplici
progetti didattici, ma nulla è come entrare nel contesto lavorativo e avere contatto diretto con questa realtà. \\
Principalmente le mie aspirazioni si possono riassumere in:
\begin{itemize}
	\item collaborare con colleghi con molta più esperienza e conoscenza di me;
	\item imparare nuove tecnologie e distinguere quelle buone da quelle cattive;
	\item migliorare la mia gestione del tempo e organizzarmi con le scadenze dei vari compiti;
	\item trovare proposte e contatti per progetti futuri.
\end{itemize}

%**************************************************************

\section{La proposta di stage}

\acrlong{asd} ha aperto una posizione per uno stage a una persona interessata al mobile development, che potesse collaborare sia nella parte di analisi, che nella parte di codifica. Il progetto era incentrato sulla formazione sulle tecnologie mobile, per fare evolvere l'azienda anche su questo ramo cercando nuove opportunità.

\subsection{Il contesto}

Come già accennato {\hyperref[cap:introduzione]{nel primo capitolo}} Alternative Studio è un'azienda giovane che da 5 anni collabora con
\gls{ucis}. Durante questi anni è stato sviluppato un gestionale e un sito per la loro organizzazione.

% chiedere info sul gestionale

All'interno di questo gestionale è stata sviluppata un'\acrshort{api}, volta alla registrazione di log contenenti la posizione tramite
coordinate e alla creazione di una relativa traccia. A questo software attualmente si interfaccia un'applicazione sviluppata 5 anni fa, non
da \acrlong{asd}, e pubblicata nel 2018 con il nome di UCIS Report Tool. L'applicazione ha come funzionalità principale la creazione e la
gestione di attività, con la registrazione e l'invio di log contenenti la posizione del dispositivo sul quale esegue. Questa, però, non
risulta essere un prodotto professionale e utilizza tecnologie già vecchie e difficilmente manutenibili o aggiornabili. Inoltre sono stati
lamentati da \gls{ucis} alcuni bug e pochissima precisione nella registrazione della posizione. Dall'associazione nasce così l'esigenza di aggiornare
o riscrivere completamente l'applicazione. \\
\noindent Alternative Studio ha colto, quindi, l'opportunità di sviluppare una sua versione dell'applicativo e di proporlo poi all'organizzazione con la quale già collabora.

\subsubsection{Problematiche dell'applicazione UCIS Report Tool}
L'applicazione attualmente utilizzata da UCIS è stata sviluppata interamente in linguaggio nativo, esclusivamente per il
sistema operativo \gls{Android}. Il principale problema rimane in ogni caso la scarsa precisione nella lettura delle
tracce GPS. L'invio dei log e troppo dilazionato nel tempo rendendo la traccia risultante frastagliata e alle volte
incomprensibile. Essendo la funzione primaria questo risulta essere un grosso problema, ma non è certamente l'unico. La
disposizione delle viste non è logica e lascia l'utente disorientato. Inoltre i form risultano essere errati: alcune
delle opzioni non devono essere presenti e la selezione di una non provoca effetti su un'altra. Ad esempio se volessi
creare una nuova attività dovrei selezionare la macrocategoria e di conseguenza filtrare la selezione della categoria
successiva. Ciò non avviene e provoca un'errore nell'invio del form. \\
\noindent Forse il problema più grande è la non manutenibilità di questo software. Il codice risulta poco modulare e la
lettura delle classi ostica. È privo di documentazione formale e di una progettazione sensata. Risulta essere un
prodotto morto e non professionale. \\
\noindent Un altro punto negativo dell'applicazione è il kit grafico utilizzato, risulta essere vecchio e di sicuro non
accattivante per l'utente e meno ancora per il cliente.

% \subsection{Obiettivi dello stage}
% \label{sec:obiettivi}

% Durante la compilazione del Piano di Lavoro, documento che mette in chiaro ciò che andrà fatto durante lo stage, con il tutor sono stati fissati gli obiettivi principali dello stage, che qui sono elencati e raggruppati come obbligatori, desiderabili e facoltativi.

% \begin{itemize}
% 	\item Obbligatori
% 	\begin{itemize}
% 		\item progettazione e realizzazione di applicazione mobile;
%     \item implementare servizio di geolocalizzazione preciso nell’applicazione;
%     \item versione beta da pubblicare sullo store.
% 	\end{itemize}

% 	\item Desiderabili
% 	\begin{itemize}
% 		\item applicazione cross-platform, combatibile anche su dispositivi datati;
%     \item messa in produzione dell’applicazione.
% 	\end{itemize}

% 	\item Facoltativi
% 	\begin{itemize}
% 		\item implementazione delle notifiche push;
%     \item integrazione di una chat;
%     \item applicazione funzionante anche su dispositivi senza servizi Google.
% 	\end{itemize}
% \end{itemize}

\subsection{Pianificazione}

In seguito si è discusso su come ripartire le ore e decidere quali processi mi avrebbero portato a soddisfare gli obiettivi sopra scritti.
Si è prodotta una pianificazione oraria che prevedeva gran parte della prima parte dello stage in formazione e creazione di una \gls{product
baseline} e una seconda parte di codifica e testing dell'applicazione, per chiudere con il collaudo e la pubblicazione sullo store Android.
\\
Il progetto di stage prevedeva circa 320 ore e in particolare si era stimata la durata delle varie attività:
\begin{itemize}
		\item \textbf{Prima Settimana}
		\begin{itemize}
				\item Analisi e formazione delle tecnologie da utilizzare nel progetto;
				\item studio del framework ionic;
				\item studio di geolocalizzazione;
				\item studio per notifiche push.
		\end{itemize}
		\item \textbf{Seconda Settimana}
		\begin{itemize}
			\item Prototipo applicazione funzionante;
			\item implementazione funzionalità di gestione delle utenze;
			\item navigazione tra schermate dell’app.
		\end{itemize}
		\item \textbf{Terza Settimana}
		\begin{itemize}
				\item implementazione registrazione delle attività.
		\end{itemize}
		\item \textbf{Quarta Settimana}
		\begin{itemize}
				\item implementazione del salvataggio offline dei dati.
		\end{itemize}
		\item \textbf{Quinta Settimana}
		\begin{itemize}
				\item aggiunta chiamante API al gestionale.
		\end{itemize}
		\item \textbf{Sesta Settimana}
		\begin{itemize}
				\item implentazione gestione emergenze.
				\item test.
		\end{itemize}
		\item \textbf{Settima Settimana}
		\begin{itemize}
				\item collaudo;
				\item messa in produzione.
		\end{itemize}
		\item \textbf{Ottava Settimana}
		\begin{itemize}
				\item aggiunta requisiti facoltativi.
		\end{itemize}
\end{itemize}

I dettagli sono riportati sul seguente \gls{diagramma di Gantt}.

\begin{figure}[h]
	\begin{center}
		\includegraphics[height=6cm]{gantt}
		\caption{\Gls{diagramma di Gantt} della ripartizione giornaliera.}
	\end{center}
\end{figure}

\subsection{Vincoli}

Questo capitolo si occupa di descrivere i vincoli ai quali sono stato sottoposto durante lo stage. Tengo a precisare che quelli imposti dall'azienda non sono mai stati un ostacolo, bensì delle guide precise per raggiungere al meglio gli obiettivi. I vincoli sono stati raggruppati nelle sezioni successive.

\subsubsection{Vincoli tecnologici}
Il vincolo più importante imposto è stato quello di non scrivere l'applicazione in linguaggio nativo per ogni sistema operativo. Nonostante
sia stata sviluppata solamente per il sistema operativo Android, l'idea iniziale era quella di produrla \gls{crossplatform} e di portarla con
poche modiche su qualunque sistema operativo mobile e non. È per questo che il tutor mi ha imposto la ricerca di un framework ibrido.

\subsubsection{Vincoli metodologici}
Il mio stage si è svolto, purtropp o, durante un periodo soggetto a restrizioni negli spostamenti dovuti all'epidemia di SARS-CoV-2. Dopo un
periodo iniziale, però, sono riuscito a svolgere lo stage in parte in presenza. Quinid ho concordato con il tutor di poter accedere ai locali di
Alternative Studio almeno due giorni a settimana, solitamente il lunedì e il venerdì. \\
Per quanto riguarda il metodo di lavoro ho cercato diverse volte consiglio presso il tutor. Riuscire a svolgere lo stage in presenza, anche
se in parte, è stato molto importante in questo senso. Infine per aderire in parte al metodo \gls{Agile} abbiamo organizzato una breve riunione
giornaliera, che aveva come scopo quello di allineare il lavoro e programmare le tasks successive.

\subsubsection{Vincoli temporali}
Oltre al vincolo di 320 ore, imposto dalla natura dello stage e dall'azienda stessa, sono dovuto sottostare alle scadenze delle varie tasks
impostate su Gitlab. In particolare si è scelto di lavorare 40 ore alla settimana. \\
Non essendo stato  uno stage completamente in presenza, non si è dato nessun vincolo sulla singola giornata lavorativa. Solitamente l'orario
lavorativo iniziava alle 9:00 e si concludeva alle 18, con un'ora di pausa pranzo. Questo per me è stato un problema da affrontare, perché
ho notato che sono molto più produttivo quando ho delle restrizioni sugli orari. Il problema è stato fatto presente al tutor che ha
collaborato nel controllo e nel conteggio delle ore effettiva di lavoro.


\section{L'azienda}

\begin{figure}[htbp]
	\begin{center}
		\includegraphics[height=6cm]{app_logo}
	\end{center}
	\caption {Logo di \acrlong{asd}.}
\end{figure}

Alternative Studio è una web agency che fornisce soluzioni professionali su misura, costruite secondo le esigenze del cliente. Si occupa
principalmente di sviluppo web e marketing. È un'azienda piccola che raccoglie poche risorse umane, ma molte energie che continuano a
spingere per crescere. Opera da appena sei anni nel settore del web development, ma ha abbracciato anche altre iniziative, collaborando in
progetti più grandi con altre realtà. Negli ultimi anni l'azienda si è cimentata nello sviluppo di un gestionale per l'organizzazione
\gls{ucis} e di una sua \acrshort{api} volta alla ricezione e all'elaborazione di attività registrate durante addestramento,
soccorso o esercitazioni.

%**************************************************************
% \section{L'idea}

% L'attuale applicazione per smartphone che si occupa di registrare le attività succitate è stata sviluppata molti anni fa ed essendo molto
% instabile è nata l'esigenza di un refactoring di quest'ultima. Volendo utilizzare tecnologie moderne per lo sviluppo mobile e crossplatform
% è nata l'esigenza di un'indagine preventiva sulle tecnologie da utilizzare. \\
% Alternative Studio ha pensato, quindi, di aprire una posizione perfetta per un laureando, che cerchi un progetto formante, che lo metta continuamente alla prova.

%**************************************************************


% \section{Strumenti utilizzati}

% Lo stage si è focalizzato sull'utilizzo di tecnologie mobile per lo sviluppo di applicazioni per smartphone. Data la veloce evoluzione di
% questo ramo, Alternative Studio ha avviato uno studio approfondito per scegliere quelle più adatte allo sviluppo della
% propria applicazione. Di seguito alcune tecnologie che siamo andati a scegliere per lo sviluppo. Lo studio approfondito e le motivazioni
% della scelta saranno spiegati in dettaglio nel \autoref{sec:studiodifattibilita}.

% \subsection{Ionic}

% Ionic è un \gls{framework} open source che fornisce strumenti agli sviluppatori, come librerie grafiche e plugins nativi per dialogare
% con \gls{Android} e \gls{iOS}. Esso permette lo sviluppo mobile utilizzando tecnologie standard web come Angular, React e Vue, evitando lo
% sviluppo in linguaggio nativo dei singoli sistemi operativi. Questo è reso possibili da wrapper esistenti su ogni piattaforma volti a
% eseguire l'applicazione.

% \subsection{Angular}

% Come framework web per lo sviluppo \gls{frontend} si è utilizzato Angular, evoluzione del noto AngularJS, sviluppato in prevalenza da Google, ma con distribuzione \gls{open source}. Le applicazioni Angular sono eseguite direttamente lato client dal web browser e quindi non vengono reinviate indietro al web server. Inoltre sono nativamente responsive, cioè i toolkit utilizzate si adattano al dispositivo sul quale sono eseguite. \\
% Nonostante l'esperenzia di Alternative Studio sull'analogo Vue, si è deciso di adottare Angular dato che Ionic-Vue era ancora in fase di
% beta e quindi potenzialmente instabile e soggetto a cambiamenti.

% \subsubsection{TypeScript}

% TypeScript è un linguaggio implementato da Microsoft nel 2012, derivato da JavaScript, al quale aggiunge il concetto di tipizzazione e di orientamento agli oggetti. Nonostante JS sia un linguaggio \gls{tipizzato}, esso non fa nessun tipo di controllo statico sui tipi effettuando sempre una conversione dinamica. Questo produce spesso degli errori difficili da trovare e correggere, per questo TypeScript introduce la compilazione che non fa altro che tradurre il codice in JavaScript, eseguendo prima un controllo dei tipi.\\
% Grazie a queste caratteristiche TypeScript non è stato utilizzato solo per il \gls{frontend}, ma anche per parte della logica
% dell'applicazione. Per questo e per il fatto che è il linguaggio adottato nativamente da Angular sarà quello più utilizzato durante lo
% sviluppo.

% \subsection{Android e Java}

% \gls{Android} è il sistema operativo mobile più diffuso nel pianeta e di proprietà di \gls{Google}. È stato scelto come riferimento durante
% lo sviluppo, nonostante Ionic offra la possibilità di sviluppare in crossplatform con lo stesso codice. Al contrario alcune
% funzionalità che si interfacciano con il sistema operativo, come ad esempio la componente GPS, devono essere sviluppate in linguaggio
% nativo, che nel caso di \gls{Android} è \gls{java}.

%**************************************************************

\subsection{Metodo di Lavoro}
Dato le contenute risorse umane a disposizione Alternative Studio adotta un ciclo di sviluppo software \gls{Incrementale} con qualche
introduzione di processi da quello \gls{Agile}, in particolared dal metodo \gls{Scrum}. Durante lo stage lo studente è stato anche
incaricato di introdurre qualche concetto delle metodologie di sviluppo moderne all'interno del contesto aziendale, come quello di
\gls{Sprint} e di \gls{Daily Stand-up}.

\subsection{Processi di sviluppo}
La mia figura è subentrata durante la fase di Analisi dei requisiti. Per questo i primi compiti affidatomi sono stati quelli di studio delle
tecnologie adatte al progetto. Durante questa fase si sono svolte alcune riunioni con il tutor, tramite videochiamata,
e redatto alcuni documenti di report riguardo le ispezioni e le ricerche. Inoltre il lavoro individuale (come compilare un "Hello World" con Ionic) si è svolto
tramite tasks. Le riunioni con il tutor sono proseguite anche durante la fase di progettazione architetturale, durante la quale abbiamo
chiarito la visione generale dell'applicazione.

\subsection{Strumenti di supporto allo sviluppo}
Ci sono alcuni software da citare utilizzati nella gestione del progetto, che \acrlong{asd} utilizza abitualmente.

\subsubsection{Gestione progetto e Versione}
Gitlab è un software open source per la gestione di repository \gls{Git} e supporto alla Continous Integration.

\begin{figure}[htbp]
	\begin{center}
		\includegraphics[height=8.5cm]{gitlab_schermata}
	\end{center}
	\caption {Schermata di nuova issue del software Gitlab.}
\end{figure}

\noindent Per la prima parte è stato importante il suo strumento di \gls{Issue Tracking System} con il quale si sono gestite le tasks e le
loro scadenze. Infatti, in questo caso le issues sono state utilizzate come metodo per tracciare i compiti da svolgere, piuttosto che per
segnalare problemi all'interno del software, come bug e affini. \\
\noindent Ogni task possiede un titolo significativo, una descrizione approfondita aggiornabile in caso di cambiamenti durante l'esecuzione
e una scadenza. Inoltre è possibile assegnarla a più membri del team, aggiungere informazioni o dialogare con il tutor attraverso la sezione
commenti e specificare la milestone alla quale è collegata.

\subsubsection{Comunicazione}
Causa il telelavoro, durante il progetto sono stati fondamentali gli strumenti di comunicazione:
\begin{itemize}
	\item Slack: per le comunicazioni ufficiali e come strumento di notifica per i cambiamenti nella repo.
	\item Telegram: come servizio di messaggistica istantanea, diretta con il tutor.
	\item Skype: per le videochiamate, essenziale per lo svolgersi delle riunioni a distanza.
\end{itemize}

\subsubsection{Codifica}
In dotazione ad Alternative Studio ho utilizzato WebStorm, potente IDE parte della suite di JetBrains. Una delle sue caratteristiche
principali sono la sua modularità, grazie allo store di plugins disponibili che aggiungono funzionalità, come il conteggio delle ore di
programmazione e gestione del repository \gls{Git} locale. \\
Inoltre si integra molto bene con le tecnologie web, come Angular, mettendo a disposizione il suo ambiente di building e di testing direttamente all'interno dell'\acrshort{ide}.

\begin{figure}[htbp]
	\begin{subfigure}{0.5\textwidth}
		\includegraphics[height=2cm]{webstorm}
		\caption{Logo di WebStorm}
	\end{subfigure}
	\begin{subfigure}{0.5\textwidth}
		\includegraphics[height=2cm]{studio}
		\caption{Logo di Android Studio}
	\end{subfigure}
\end{figure}

\subsubsection{Android Studio}
Android Studio fa parte sempre del pacchetto IDEA di JetBrains ed è una versione molto ridotta di Intellij. A differenza del suo
fratello maggiore è open source e ottimizzato per lo sviluppo Android. I linguaggi nativi utilizzabili in questo \acrshort{ide} sono principalmente
\gls{Go} e \gls{java}. Utilizza gradle come strumento automatico per la build dei progetti ed è estensibile come le altre applicazioni IDEA
tramite plugins.

\subsubsection{Postman}
Programma a supporto della codifica, il quale mi ha aiutato nell'interfacciarmi alle API. Postman permette di fare chiamate \acrshort{http} ad
un'API e di visualizzarne il risultato. È un software molto utile nel quale puoi, ad esempio, testare le stringhe che utilizzerai nel tuo
codice, o per analizzare il risultato per capire come utilizzarlo.

%**************************************************************

\section{Convenzioni tipografiche}

Riguardo la stesura del testo, relativamente al documento sono state adottate le seguenti convenzioni tipografiche:
\begin{itemize}
	\item gli acronimi, le abbreviazioni e i termini ambigui o di uso non comune menzionati vengono definiti nel glossario, situato alla fine del presente documento;
	\item per la prima occorrenza dei termini riportati nel glossario viene utilizzata la seguente nomenclatura: \emph{parola}\glsfirstoccur;
	\item i termini in lingua straniera o facenti parti del gergo tecnico sono evidenziati con il carattere \emph{corsivo}.
\end{itemize}

\section{Scopo del documento}
In questo documento andrò a descrivere la mia esperienza di stage, raccontando la mia preparazione, lo sviluppo e le mie valutazioni
riguardo ad esso. Inoltre descriverò in modo approfondito le tecnologie che sono andato ad analizzare e studiare e le mie opinioni su di esse.


\subsection{Organizzazione del testo}

\begin{description}

    \item[{\hyperref[cap:analisi-requisiti]{Il secondo capitolo}}] approfondisce i contenuti del progetto di stage e i motivi della scelta.

    \item[{\hyperref[cap:progettazione-realizzazione]{Il terzo capitolo}}] descrive in modo dettagliato ciò che è stato fatto durante lo stage.

    \item[{\hyperref[cap:valutazione-retrospettiva]{Il quarto capitolo}}] contiene le considerazioni a posteriori dell'esperienza di stage.

\end{description}

% Introduzione al contesto applicativo.\\
% \noindent Esempio di utilizzo di un termine nel glossario \\
% \gls{api}. \\
% \noindent Esempio di citazione in linea \\
% \cite{site:agile-manifesto}. \\
% \noindent Esempio di citazione nel pie' di pagina \\
% citazione\footcite{womak:lean-thinking} \\

%**************************************************************
