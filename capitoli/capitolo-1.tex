% !TEX encoding = UTF-8
% !TEX TS-program = pdflatex
% !TEX root = ../tesi.tex

%**************************************************************
\chapter{Introduzione}
\label{cap:introduzione}
%**************************************************************

\section{L'azienda}

\begin{figure}[htbp]
	\begin{center}
		\includegraphics[height=6cm]{app_logo}
	\end{center}
	\caption {Logo di \acrlong{asd}.}
\end{figure}

Alternative Studio è una web agency che fornisce soluzioni professionali su misura, costruite secondo le esigenze del cliente. Si occupa
principalmente di sviluppo web e marketing. È un'azienda piccola che raccoglie poche risorse umane, ma molte energie che continuano a
spingere per crescere. Opera da appena sei anni nel settore del web development, ma ha abbracciato anche altre iniziative, collaborando in
progetti più grandi con altre realtà. Negli ultimi anni l'azienda si è cimentata nello sviluppo di un gestionale per l'organizzazione
\gls{ucis} e di una sua \acrshort{api} volta alla ricezione e all'elaborazione di attività registrate durante addestramento,
soccorso o esercitazioni.

%**************************************************************
\section{L'idea}

L'attuale applicazione per smartphone che si occupa di registrare le attività succitate è stata sviluppata molti anni fa ed essendo molto
instabile è nata l'esigenza di un refactoring di quest'ultima. Volendo utilizzare tecnologie moderne per lo sviluppo mobile e crossplatform
è nata l'esigenza di un'indagine preventiva sulle tecnologie da utilizzare. \\
Alternative Studio ha pensato, quindi, di aprire una posizione perfetta per un laureando, che cerchi un progetto formante, che lo metta continuamente alla prova.

%**************************************************************


% \section{Strumenti utilizzati}

% Lo stage si è focalizzato sull'utilizzo di tecnologie mobile per lo sviluppo di applicazioni per smartphone. Data la veloce evoluzione di
% questo ramo, Alternative Studio ha avviato uno studio approfondito per scegliere quelle più adatte allo sviluppo della
% propria applicazione. Di seguito alcune tecnologie che siamo andati a scegliere per lo sviluppo. Lo studio approfondito e le motivazioni
% della scelta saranno spiegati in dettaglio nel \autoref{sec:studiodifattibilita}.

% \subsection{Ionic}

% Ionic è un \gls{framework} open source che fornisce strumenti agli sviluppatori, come librerie grafiche e plugins nativi per dialogare
% con \gls{Android} e \gls{iOS}. Esso permette lo sviluppo mobile utilizzando tecnologie standard web come Angular, React e Vue, evitando lo
% sviluppo in linguaggio nativo dei singoli sistemi operativi. Questo è reso possibili da wrapper esistenti su ogni piattaforma volti a
% eseguire l'applicazione.

% \subsection{Angular}

% Come framework web per lo sviluppo \gls{frontend} si è utilizzato Angular, evoluzione del noto AngularJS, sviluppato in prevalenza da Google, ma con distribuzione \gls{open source}. Le applicazioni Angular sono eseguite direttamente lato client dal web browser e quindi non vengono reinviate indietro al web server. Inoltre sono nativamente responsive, cioè i toolkit utilizzate si adattano al dispositivo sul quale sono eseguite. \\
% Nonostante l'esperenzia di Alternative Studio sull'analogo Vue, si è deciso di adottare Angular dato che Ionic-Vue era ancora in fase di
% beta e quindi potenzialmente instabile e soggetto a cambiamenti.

% \subsubsection{TypeScript}

% TypeScript è un linguaggio implementato da Microsoft nel 2012, derivato da JavaScript, al quale aggiunge il concetto di tipizzazione e di orientamento agli oggetti. Nonostante JS sia un linguaggio \gls{tipizzato}, esso non fa nessun tipo di controllo statico sui tipi effettuando sempre una conversione dinamica. Questo produce spesso degli errori difficili da trovare e correggere, per questo TypeScript introduce la compilazione che non fa altro che tradurre il codice in JavaScript, eseguendo prima un controllo dei tipi.\\
% Grazie a queste caratteristiche TypeScript non è stato utilizzato solo per il \gls{frontend}, ma anche per parte della logica
% dell'applicazione. Per questo e per il fatto che è il linguaggio adottato nativamente da Angular sarà quello più utilizzato durante lo
% sviluppo.

% \subsection{Android e Java}

% \gls{Android} è il sistema operativo mobile più diffuso nel pianeta e di proprietà di \gls{Google}. È stato scelto come riferimento durante
% lo sviluppo, nonostante Ionic offra la possibilità di sviluppare in crossplatform con lo stesso codice. Al contrario alcune
% funzionalità che si interfacciano con il sistema operativo, come ad esempio la componente GPS, devono essere sviluppate in linguaggio
% nativo, che nel caso di \gls{Android} è \gls{java}.

%**************************************************************

\section{Metodo di Lavoro}
Dato le contenute risorse umane a disposizione Alternative Studio adotta un ciclo di sviluppo software \gls{Incrementale} con qualche
introduzione di processi da quello \gls{Agile}, in particolared dal metodo \gls{Scrum}. Durante lo stage lo studente è stato anche
incaricato di introdurre qualche concetto delle metodologie di sviluppo moderne all'interno del contesto aziendale, come quello di
\gls{Sprint} e di \gls{Daily Stand-up}.

\subsection{Processi di sviluppo}
La mia figura è subentrata durante la fase di Analisi dei requisiti. Per questo i primi compiti affidatomi sono stati quelli di studio delle
tecnologie adatte al progetto. Durante questa fase si sono svolte alcune riunioni con il tutor, tramite videochiamata,
e redatto alcuni documenti di report riguardo le ispezioni e le ricerche. Inoltre il lavoro individuale (come compilare un "Hello World" con Ionic) si è svolto
tramite tasks. Le riunioni con il tutor sono proseguite anche durante la fase di progettazione architetturale, durante la quale abbiamo
chiarito la visione generale dell'applicazione.

\subsection{Strumenti di supporto allo sviluppo}
Ci sono alcuni software da citare utilizzati nella gestione del progetto, che \acrlong{asd} utilizza abitualmente.

\subsubsection{Gestione progetto e Versione}
Gitlab è un software open source per la gestione di repository \gls{Git} e supporto alla Continous Integration.

\begin{figure}[htbp]
	\begin{center}
		\includegraphics[height=8.5cm]{gitlab_schermata}
	\end{center}
	\caption {Schermata di nuova issue del software Gitlab.}
\end{figure}

\noindent Per la prima parte è stato importante il suo strumento di \gls{Issue Tracking System} con il quale si sono gestite le tasks e le
loro scadenze. Infatti, in questo caso le issues sono state utilizzate come metodo per tracciare i compiti da svolgere, piuttosto che per
segnalare problemi all'interno del software, come bug e affini. \\
\noindent Ogni task possiede un titolo significativo, una descrizione approfondita aggiornabile in caso di cambiamenti durante l'esecuzione
e una scadenza. Inoltre è possibile assegnarla a più membri del team, aggiungere informazioni o dialogare con il tutor attraverso la sezione
commenti e specificare la milestone alla quale è collegata.

\subsubsection{Comunicazione}
Causa il telelavoro, durante il progetto sono stati fondamentali gli strumenti di comunicazione:
\begin{itemize}
	\item Slack: per le comunicazioni ufficiali e come strumento di notifica per i cambiamenti nella repo.
	\item Telegram: come servizio di messaggistica istantanea, diretta con il tutor.
	\item Skype: per le videochiamate, essenziale per lo svolgersi delle riunioni a distanza.
\end{itemize}

\subsubsection{Codifica}
In dotazione ad Alternative Studio ho utilizzato WebStorm, potente IDE parte della suite di JetBrains. Una delle sue caratteristiche
principali sono la sua modularità, grazie allo store di plugins disponibili che aggiungono funzionalità, come il conteggio delle ore di
programmazione e gestione del repository \gls{Git} locale. \\
Inoltre si integra molto bene con le tecnologie web, come Angular, mettendo a disposizione il suo ambiente di building e di testing direttamente all'interno dell'\acrshort{ide}.

\begin{figure}[htbp]
	\begin{subfigure}{0.5\textwidth}
		\includegraphics[height=2cm]{webstorm}
		\caption{Logo di WebStorm}
	\end{subfigure}
	\begin{subfigure}{0.5\textwidth}
		\includegraphics[height=2cm]{studio}
		\caption{Logo di Android Studio}
	\end{subfigure}
\end{figure}

\subsubsection{Android Studio}
Android Studio fa parte sempre del pacchetto IDEA di JetBrains ed è una versione molto ridotta di Intellij. A differenza del suo
fratello maggiore è open source e ottimizzato per lo sviluppo Android. I linguaggi nativi utilizzabili in questo \acrshort{ide} sono principalmente
\gls{Go} e \gls{java}. Utilizza gradle come strumento automatico per la build dei progetti ed è estensibile come le altre applicazioni IDEA
tramite plugins.

\subsubsection{Postman}
Programma a supporto della codifica, il quale mi ha aiutato nell'interfacciarmi alle API. Postman permette di fare chiamate \acrshort{http} ad
un'API e di visualizzarne il risultato. È un software molto utile nel quale puoi, ad esempio, testare le stringhe che utilizzerai nel tuo
codice, o per analizzare il risultato per capire come utilizzarlo.

%**************************************************************

\section{Convenzioni tipografiche}

Riguardo la stesura del testo, relativamente al documento sono state adottate le seguenti convenzioni tipografiche:
\begin{itemize}
	\item gli acronimi, le abbreviazioni e i termini ambigui o di uso non comune menzionati vengono definiti nel glossario, situato alla fine del presente documento;
	\item per la prima occorrenza dei termini riportati nel glossario viene utilizzata la seguente nomenclatura: \emph{parola}\glsfirstoccur;
	\item i termini in lingua straniera o facenti parti del gergo tecnico sono evidenziati con il carattere \emph{corsivo}.
\end{itemize}

\section{Scopo del documento}
In questo documento andrò a descrivere la mia esperienza di stage, raccontando la mia preparazione, lo sviluppo e le mie valutazioni
riguardo ad esso. Inoltre descriverò in modo approfondito le tecnologie che sono andato ad analizzare e studiare e le mie opinioni su di esse.


\subsection{Organizzazione del testo}

\begin{description}

    \item[{\hyperref[cap:descrizione-stage]{Il secondo capitolo}}] approfondisce i contenuti del progetto di stage e i motivi della scelta.

    \item[{\hyperref[cap:ilprogetto]{Il terzo capitolo}}] descrive in modo dettagliato ciò che è stato fatto durante lo stage.

    \item[{\hyperref[cap:valutazione-retrospettiva]{Il quarto capitolo}}] contiene le considerazioni a posteriori dell'esperienza di stage.

\end{description}

% Introduzione al contesto applicativo.\\
% \noindent Esempio di utilizzo di un termine nel glossario \\
% \gls{api}. \\
% \noindent Esempio di citazione in linea \\
% \cite{site:agile-manifesto}. \\
% \noindent Esempio di citazione nel pie' di pagina \\
% citazione\footcite{womak:lean-thinking} \\

%**************************************************************
