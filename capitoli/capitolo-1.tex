% !TEX encoding = UTF-8
% !TEX TS-program = pdflatex
% !TEX root = ../tesi.tex

%**************************************************************
\chapter{Introduzione}
\label{cap:introduzione}
%**************************************************************

% Introduzione al contesto applicativo.\\
% \noindent Esempio di utilizzo di un termine nel glossario \\
% \gls{api}. \\
% \noindent Esempio di citazione in linea \\
% \cite{site:agile-manifesto}. \\
% \noindent Esempio di citazione nel pie' di pagina \\
% citazione\footcite{womak:lean-thinking} \\

%**************************************************************
\section{L'azienda}

\begin{figure}[htbp]
\begin{center}
\includegraphics[height=6cm]{app_logo}
\end{center}
\end{figure}

Alternative Studio è una web agency che fornisce soluzioni professionali su misura, costruite secondo le esigenze del cliente. Si occupa principalmente di sviluppo web e marketing. È un ambiente piccolo, ma in veloce sviluppo, orientato alla formazione e all'utilizzo di tecnologie nuove. \\
Negli ultimi anni l'azienda si è cimentata nello sviluppo di un gestionale per l'organizzazione \gls{ucis}. Quest ultimo possiede un' \gls{api} volta alla ricezione e all'elaborazione di attività registrate durante addestramento, soccorso o esercitazioni.

%**************************************************************
\section{L'idea}

L'attuale applicazione per smartphone che si occupa di registrare le attività succitate è stata sviluppata molti anni fa ed essendo molto instabile è nata l'esigenza di un refactoring di quest'ultima. Volendo utilizzare tecnologie moderne per lo sviluppo mobile e crossplatform è nata l'esigenza di un'indagine preventiva sulle tecnologie da utilizzare. \\
Alternativer Studio ha pensato, quindi, di aprire una posizione perfetta per un laureando, che cerchi un progetto formante, che lo metta continuamente alla prova.

%**************************************************************
\section{UCIS Report Tool}

In questa sezione verrano descritti alcuni dei problemi dell'attuale applicazione in dotazione a \gls{ucis} e i motivi per i quali si è deciso di ristrutturarla.
L'applicazione \gls{UCIS Report Tool} presente ancora nello store Android, è stata sviluppata nel framework nativo di Android utilizzando \gls{Java} e \gls{xml}.
L'interfaccia è datata e non adatta al target degli utenti.

%**************************************************************
\section{Organizzazione del testo}

% \begin{description}
%     \item[{\hyperref[cap:processi-metodologie]{Il secondo capitolo}}] descrive ...
%
%     \item[{\hyperref[cap:descrizione-stage]{Il terzo capitolo}}] approfondisce ...
%
%     \item[{\hyperref[cap:analisi-requisiti]{Il quarto capitolo}}] approfondisce ...
%
%     \item[{\hyperref[cap:progettazione-codifica]{Il quinto capitolo}}] approfondisce ...
%
%     \item[{\hyperref[cap:verifica-validazione]{Il sesto capitolo}}] approfondisce ...
%
%     \item[{\hyperref[cap:conclusioni]{Nel settimo capitolo}}] descrive ...
% \end{description}

Riguardo la stesura del testo, relativamente al documento sono state adottate le seguenti convenzioni tipografiche:
\begin{itemize}
	\item gli acronimi, le abbreviazioni e i termini ambigui o di uso non comune menzionati vengono definiti nel glossario, situato alla fine del presente documento;
	\item per la prima occorrenza dei termini riportati nel glossario viene utilizzata la seguente nomenclatura: \emph{parola}\glsfirstoccur;
	\item i termini in lingua straniera o facenti parti del gergo tecnico sono evidenziati con il carattere \emph{corsivo}.
\end{itemize}
