% !TEX encoding = UTF-8
% !TEX TS-program = pdflatex
% !TEX root = ../tesi.tex

%**************************************************************
\chapter{Descrizione dello Stage}
\label{cap:descrizione-stage}
%**************************************************************

% \intro{Brevissima introduzione al capitolo}

%**************************************************************
\section{Scelta dello Stage}

La scelta dello stage è stata molto difficile. Ero alla ricerca, infatti, di un contesto specifico per la mia prima esperienza lavorativa nel settore informatico. Consultando tutte le varie proposte, durante l'evento Stage-IT, ho fatto fatica a trovare una proposta che mi convincesse a fondo. Alcune delle aziende che ho contattato si sono dimostrate disponibili nei miei confronti, ma per tempistiche legate alla laurea non siamo riusciti a venirci incontro. Tra le motivazioni per le quali ho scelto \gls{asd} c'è senz'altro un progetto chiaro e stimolante, ma anche la disponibilità immediata allo stage.

\subsection{Motivazioni della scelta}

Quello di cui ero alla ricerca, come già accennato, era un contesto piccolo con ampi margini di crescita e di formazione personale sulle tecnologie moderne, più che sul metodo di lavoro, che invece tende a essere molto più definito in aziende più grandi, con molti dipendenti. Quello che mi interessava era sperimentare vari ruoli, comprendere le varie responsabilità che questi comportano e come interfacciarsi con altri dipendenti. \\
Oltre al contesto aziendale ero attratto anche dall'argomento del progetto. Volevo sperimentare nuove tecnologie e nuovi linguaggi, lavorare su software nuovo, piuttosto che su manutenzione o aggiornamento di software vecchio. Da questo punto di vista l'Università di Padova, attraverso Stage-IT mi ha messo in contatto con molte aziende interessanti.
La mia ricerca del progetto di Stage si è basata sul desiderio di lavorare in ambito frontend e questo ha ristretto di molto le varie opportunità. \\
Un altro filtro molto importante al quale tengo molto è il fatto che ho sempre voluto lavorare con il lato \gls{frontend}, al quale credo di essere predisposto. \\
Il fattore decisivo sulla scelta dello stage è stato il fatto che da tempo volevo sperimentare le tecnologie per lo sviluppo di applicazioni mobile. Io stesso, precedentemente allo stage, avevo provato ad avviare qualche piccolo progetto in questo campo e per questo ho voluto scegliere un progetto che avesse questo come tema principale. \\
In questo senso la proposta di \acrlong{asd} ha subito catturato la mia attenzione e nonostante fosse esterna, ho preso contatti per avviare con loro il mio stage.

\subsection{Obiettivi personali}
\label{sec:obiettivi-pers}
Le esperienze lavorative precedenti a questa mi avevano insegnato cosa vuol dire lavorare in una squadra e cosa vuol dire prendersi carico di responsabilità, ma con nessuna di queste avevo messo in pratica ciò che avevo studiato finora. Sentivo quindi il bisogno di consolidare il bagaglio di nozioni acquisite durante gli anni del corso e lo stage ne è stata l'occasione perfetta. Certo, ci sono stati i molteplici progetti didattici, ma nulla è come entrare nel contesto lavorativo e avere contatto diretto con questa. \\
Principalmente le mie aspirazioni si possono riassumere in:
\begin{itemize}
	\item collaborare con colleghi con molta più esperienza e conoscenza di me;
	\item imparare nuove tecnologie e distinguere quelle buone da quelle cattive;
	\item migliorare la mia gestione del tempo e organizzarmi con le scadenze dei vari compiti;
	\item trovare proposte e contatti per progetti futuri.
\end{itemize}

%**************************************************************

\section{La proposta di stage}

\gls{asd} ha aperto una posizione per uno stage a una persona interessata al mobile development, che potesse collaborare sia nella parte di analisi, che nella parte di codifica. Il progetto era incentrato sulla formazione sulle tecnologie mobile, per fare evolvere l'azienda anche su questo ramo cercando nuove opportunità.

\subsection{Il contesto}

Come già accennato {\hyperref[cap:introduzione]{nel primo capitolo}} Alternative Studio è un'azienda giovane che da 5 anni collabora con \gls{ucis}. Durante questi anni è stato sviluppato un gestionale e un sito per la loro organizzazione.

% chiedere info sul gestionale

All'interno di questo gestionale è stata sviluppata un'\gls{api}, volta alla registrazione di log contenenti la posizione tramite coordinate e alla creazione di una relativa traccia. Al gestione attualmente si interfaccia un'applicazione sviluppata 5 anni fa da un privato

% correggere il numero di anni

e pubblicata nel 2018 con il nome di UCIS Report Tool. L'applicazione ha come funzionalità principale la creazione e la gestione di attività, con la registrazione e l'invio di log all'\gls{api} dedicata. Questa, però, non risulta essere un prodotto professionale e utilizza tecnologie già vecchie e difficilmente manutenibili o aggiornabili. Inoltre sono stati lamentati da \gls{ucis} alcuni bug e pochissima precisione nella registrazione della posizione. Nasce così l'esigenza di aggiornare l'applicazione da parte di \gls{ucis}. \\
\noindent Alternative Studio ha colto, quindi, l'opportunità di scrivere una sua versione dell'applicativo e di proporlo poi all'organizzazione con la quale già collabora.

\subsection{Obiettivi dello stage}
\label{sec:obiettivi}

Durante la compilazione del Piano di Lavoro, documento che mette in chiaro ciò che andrà fatto durante lo stage, con il tutor sono stati fissati gli obiettivi principali dello stage, che qui sono elencati e raggruppati come obbligatori, desiderabili e facoltativi.

\begin{itemize}
	\item Obbligatori
	\begin{itemize}
		\item progettazione e realizzazione di applicazione mobile;
    \item implementare servizio di geolocalizzazione preciso nell’applicazione;
    \item versione beta da pubblicare sullo store.
	\end{itemize}

	\item Desiderabili
	\begin{itemize}
		\item applicazione cross-platform, combatibile anche su dispositivi datati;
    \item messa in produzione dell’applicazione.
	\end{itemize}

	\item Facoltativi
	\begin{itemize}
		\item implementazione delle notifiche push;
    \item integrazione di una chat;
    \item applicazione funzionante anche su dispositivi senza servizi Google.
	\end{itemize}
\end{itemize}

\subsection{Pianificazione}

In seguito si è discusso su come ripartire le ore e decidere quali processi mi avrebbero portato a soddisfare gli obiettivi sopra scritti. Si è prodotta una pianificazione oraria che prevedeva gran parte della prima parte dello stage in formazione e creazione di una \gls{product baseline} e una seconda parte di codifica e testing dell'applicazione, per chiudere con il collaudo e la pubblicazione sullo store Android. \\
Il progetto di stage prevedeva circa 320 ore e in particolare si era stimata la durata delle varie attività:
\begin{itemize}
		\item \textbf{Prima Settimana}
		\begin{itemize}
				\item Analisi e formazione delle tecnologie da utilizzare nel progetto;
				\item studio del framework ionic;
				\item studio di geolocalizzazione;
				\item studio per notifiche push.
		\end{itemize}
		\item \textbf{Seconda Settimana}
		\begin{itemize}
			\item Prototipo applicazione funzionante;
			\item implementazione funzionalità di gestione delle utenze;
			\item navigazione tra schermate dell’app.
		\end{itemize}
		\item \textbf{Terza Settimana}
		\begin{itemize}
				\item implementazione registrazione delle attività.
		\end{itemize}
		\item \textbf{Quarta Settimana}
		\begin{itemize}
				\item implementazione del salvataggio offline dei dati.
		\end{itemize}
		\item \textbf{Quinta Settimana}
		\begin{itemize}
				\item aggiunta chiamante API al gestionale.
		\end{itemize}
		\item \textbf{Sesta Settimana}
		\begin{itemize}
				\item implentazione gestione emergenze.
				\item test.
		\end{itemize}
		\item \textbf{Settima Settimana}
		\begin{itemize}
				\item collaudo;
				\item messa in produzione.
		\end{itemize}
		\item \textbf{Ottava Settimana}
		\begin{itemize}
				\item aggiunta requisiti facoltativi.
		\end{itemize}
\end{itemize}

I dettagli sono riportati sul seguente \gls{diagramma di Gantt}.

\begin{figure}[h]
	\begin{center}
		\includegraphics[height=6cm]{gantt}
		\caption{\gls{diagramma di Gantt} della ripartizione giornaliera.}
	\end{center}
\end{figure}

\subsection{Vincoli}

Questo capitolo si occupa di descrivere i vincoli ai quali sono stato sottoposto durante lo stage. Tengo a precisare che quelli imposti dall'azienda non sono mai stati un ostacolo, bensì delle guide precise per raggiungere al meglio gli obiettivi. I vincoli sono stati raggruppati nelle sezioni successive.

\subsubsection{Vincoli tecnologici}
Il vincolo più importante imposto è stato quello di non scrivere l'applicazione in linguaggio nativo per ogni sistema operativo. Nonostante sia stata sviluppata solamente per il sistema operativo Android, l'idea iniziale era quella di farla \gls{crossplatform} e di portarla con poche modiche su qualunque sistema operativo mobile e non. È per questo che il tutor mi ha imposto la ricerca di un framework crossplatform. I linguaggi e le tecnologie sono venute di conseguenza: in questo ambito si utilizza per forza un framework web, che in questo caso è stato scelto in comune accordo, e TypeScript a supporto di esso.

\subsubsection{Vincoli metodologici}
Il mio stage si è svolto, purtroppo, durante un periodo soggetto a restrizioni negli spostamenti dovuti all'epidemia di SARS-CoV-2. Dopo un periodo iniziale, infatti sono riuscito a svolgere lo stage in parte in presenza. Ho concordato con il tutor di poter accedere ai locali di Alternative Studio almeno due giorni a settimana, solitamente il lunedì e il venerdì. \\
Per quanto riguarda il metodo di lavoro ho cercato diverse volte consiglio presso il tutor. Riuscire a svolgere lo stage in presenza, anche se in parte, è stato molto importante in questo senso. Quello che cercavo infatti era il confronto continuo, per cercare di imparare e di crescere. Quello che mi è stato chiesto è stata la giornaliera riunione, per mettere in chiaro gli obiettivi giornalieri e la direzione che stava prendendo il team.

\subsubsection{Vincoli temporali}
Oltre al vincolo di 320 ore, imposto dalla natura dello stage e dall'azienda stessa, sono dovuto sottostare alle scadenze delle varie tasks impostate su Gitlab. In particolare si è scelto di lavorare 40 ore alla settimana. \\
Non essendo stato uno stage completamente in presenza, non si è dato nessun vincolo sulla singola giornata lavorativa. Solitamente l'orario lavorativo iniziava alle 9:00 e si concludeva alle 18, con un'ora di pausa pranzo. Questo per me è stato un problema da affrontare, perché ho notato che sono molto più produttivo quando ho delle restrizioni sugli orari. Il problema è stato fatto presente al tutor che ha collaborato nel controllo e nel conteggio delle ore effettiva di lavoro.
