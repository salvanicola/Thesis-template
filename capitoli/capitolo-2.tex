% !TEX encoding = UTF-8
% !TEX TS-program = pdflatex
% !TEX root = ../tesi.tex

%**************************************************************
\chapter{Processi e metodologie}
\label{cap:processi-metodologie}
%**************************************************************

\intro{Brevissima introduzione al capitolo}

%**************************************************************
\section{Processo sviluppo prodotto}

\section{Metodo di Lavoro}

Dato le contenute risorse umane a disposizione Alternative Studio adotta un ciclo di sviluppo software \gls{Incrementale} con qualche introduzione di processi da quello \gls{Agile}, in particolared dal metodo \gls{Scrum}. Durante lo stage lo studente è stato incaricato anche di introdurre qualche concetto di questi cicli di vita all'interno del contesto aziendale, come quello di \gls{Sprint} e di \gls{Daily Scrum}. \\
\noindent A supporto dello sviluppo del progetto sono stati utilizzati i seguenti strumenti:
\begin{itemize}
	\item Gitlab: software open source per la gestione di repository \gls{Git} e supporto alla Continous Integration.
	\begin{figure}[htbp]
	\begin{center}
	\includegraphics[height=6cm]{gitlab}
	\end{center}
	\end{figure}
  Utile anche per l'assegnazione di piccole tasks e fissare scadenze, milestone e monitoraggio delle statistiche del progetto.
  \item Slack:
  \item Telegram:
  \item WebStorm: potente IDE, modulare, che con l'aggiunta di plugin acquista funzionalità, come il conteggio delle ore di programmazione, scadenze e gestione del repository \gls{Git} locale.
  \begin{figure}[htbp]
  \begin{center}
  \includegraphics[height=6cm]{webstorm}
  \end{center}
  \end{figure}
\end{itemize}
